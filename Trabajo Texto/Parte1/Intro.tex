John Conway naci\'o el 26 de diciembre de 1937, en Liverpool, Inglaterra. En los a\~nos 1970 se di\'o a conocer en la comunidad de matem\'aticas puras al encontrar tres grupos simples que nunca antes se hab\'ian encontrado, a estos ahora se les conoce por el nombre de \textit{Conway groups}, sin embargo, para la comunidad general de gente interesada en matem\'atica sus aportes m\'as destacables no fueron esos.

Desde su vida temprana demostr\'o un gran inter\'es por una parte de la matem\'atica en espec\'ifico, la matem\'atica recreativa. Conway a los quince desarroll\'o un algoritmo para calcular dada cualquier fecha en qu\'e d\'ia de la semana ca\'ia, mientras cursaba su pregrado hizo una maquina que sumaba n\'umeros y solamente funcionaba con agua, y luego del pregrado, junto a Michael Guy, fue el primero en clasificar todos los politopos uniformes de cuatro dimensiones \cite{Roberts2015-ur}.

Pero problablemente sus aportes m\'as destacados en la matem\'atica recreativa fueron sus colaboraciones con Martin Gardner. Martin Gardner fue un divulgador cient\'ifico, y tambi\'en mago, que ten\'ia una columna en el \textit{Scientific American} dedicada casi exclusivamente a proponer retos y a divulgar la matem\'atica, su columna titulada \textit{Mathematical Games} influenci\'o a muchos matem\'aticos y matem\'aticas en Estados Unidos, y en el mundo \cite{Gardner1999}.

En esta secci\'on vamos a hacer un repaso de un par de temas que surgieron a partir de esta interacci\'on.