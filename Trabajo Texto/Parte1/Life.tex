\section{El juego de la vida}

En Octubre de 1970, Martin Gardner en su columna \textit{Mathematical Games} del \textit{Scientific American} \cite{Gardner1970} escribi\'o un texto que cambi\'o para siempre la vida de Conway. El escrito \textit{The fantastic combinations of John Conway's new solitaire game ``life''}, describe un autómata celular en dos dimensiones, una suerte de juego con cero jugadores tal que, a partir de un patr\'on inicial, evolucionaba conforme a unas reglas sencillas.

La columna, adem\'as de describir el juego propon\'ia un reto: encontrar un patr\'on inicial que al evolucionar creciera sin l\'imites, y adem\'as promet\'ia 50 USD a quien fuera el primero en encontrarla o demostrar que no existe.

La columna fue un \'exito, el juego de la vida se volvi\'o un cl\'asico de culto. Hoy en d\'ia si se busca el juego de la vida de Conway en Google la p\'agina misma empieza a jugar el juego de la vida, y si por ejemplo, se busca el juego de la vida en Youtube se puede encontrar much\'isimos ejemplos de creaciones asociadas al juego de la vida, calculadoras en el juego de la vida, relojes, otros juegos parecidos al juego de la vida, incluso hay videos como el que se encuentra en \cite{YTEpicLife} que muestran `naves espaciales' y otros `organismos' viviendo en el juego. Como describir\'ia Siobhan Roberts en la biograf\'ia de Conway \cite{Roberts2015-ur}:

\textit{``Hablando de manera pr\'actica, el juego (de la vida) empuj\'o el uso de autómatas celulares y simulaciones basadas en agentes en las ciencias de la complejidad, modelando el comportamiento de todo, desde hormigas pasando por el tr\'afico, por las nubes y hasta las galaxias. Hablando de manera poco pr\'actica, el juego se convirti\'o en un cl\'asico de culto para aquellos entusiasmados en ninguna otra aplicaci\'on extravagante m\'as que perder el tiempo.'' \footnote{Traducci\'on del autor.}}

En la biograf\'ia tambi\'en cuentan varias leyendas que se cuentan alrededor del juego de la vida. Se dice que un informe militar de Estados Unidos estim\'o que el tiempo total gastado por personas jugando el juego de la vida en el trabajo equival\'ia a millones de dólares perdidos, tambi\'en se dice que en el momento en que el juego fue m\'as viral, $\frac{1}{4}$ de los computadores del mundo se jugaba el juego de la vida.

Fue tanta la fama del juego de la vida que su creador, John Horton Conway, lleg\'o a decir que odiaba el juego de la vida ya que cada vez que se mencionaba su nombre en algún art\'iculo matem\'atico lo \'unico que se mencionaba era el juego de la vida, casi que eclipsando todos los otros logros matem\'aticos que \'el hab\'ia hecho \cite{YTNumberphileConway}.

El juego de la vida es un aut\'omata celular de dos dimensiones, pero, ¿esto qu\'e significa? La parte de aut\'omata celular significa que es un juego que se juega en casillas, las casillas tienen estados posibles y estas van cambiando cada tiempo seg\'un unas reglas bien definidas. La parte de dos dimensiones significa que las casillas forman una red dos-dimensional, que en el caso del juego de la vida es una cuadr\'icula. En la secci\'on \ref{subsection:life-explain} vamos a ahondar en los detalles del juego.
\subsection{Los inicios}

La idea del juego de la vida comenz\'o desde mucho antes. Seg\'un Stephen Wolfram \cite{MathematicalArtist2022}, Conway le cont\'o que en ese tiempo hab\'ia sido contratado como profesor de l\'ogica pero que ese no era su campo de investigaci\'on principal. Conway se interes\'o entonces en los aut\'omatas, su plan era encontrar una buena enumeraci\'on de las funciones recursivas. En ese entonces Conway ten\'ia una copia del libro \textit{Automata Studies} \cite{Shannon1956}, una recolecci\'on de ensayos sobre aut\'omatas recopilados por Claude Shannon y John McCarthy, los padres de la teor\'ia de la informaci\'on y de la inteligencia artificial respectivamente. Conway cuenta que fueron las ideas de c\'omo los aut\'omatas podr\'ian en alg\'un momento simular cosas complejas como el cerebro humano, o incluso, podr\'ian replicarse a s\'i mismas, las que dieron inicio a su inter\'es en crear aut\'omatas. John von Neumann por ejemplo, ve\'ia un potencial en los aut\'omatas inmenso, pensaba en que en alg\'un momento se podr\'ia crear una m\'aquina que pudiera crear otras m\'aquinas y utilizarlas colonizando otros planetas. La idea que Von Neumann desarrollaba, ya m\'as abstra\'ida al terreno matem\'atico, era sobre un aut\'omata celular dos dimensional con la posibilidad de replicarse a s\'i mismo, la idea fue publicada despu\'es de su muerte en su libro \textit{Theory of self-reproducing automata} \cite{Von_Neumann1967-ma}. Sin embargo Conway le ve\'ia un problema al resultado de Von Neumann, cada celda del aut\'omata podr\'ia estar en un total de 29 estados, lo que \'el consideraba muy complejo. Como dir\'ia \'el en su biograf\'ia:

\textit{``\dots Parec\'ia terriblemente complicado. Lo que me emociona son las cosas que tienen una maravillosa simplicidad.''} \footnote{Traducci\'on del autor.}

Y as\'i empez\'o una busqueda hac\'ia un aut\'omata universal (que pueda replicar cualquier m\'aquina), y adem\'as que fuera lo m\'as simple posible. Algo que describir\'ia como su \textit{Jugendtraum}, su sueño de juventud.

En principio un aut\'omata es una abstracci\'on de lo que es una m\'aquina. B\'asicamente, un aut\'omata es un conjunto de estados, que representar\'ian los estados en los que puede estar la m\'aquina, un conjunto de reglas o instrucciones sobre qu\'e hacer en cada estado, y un conjunto de posibles \textit{inputs} o entradas que se le pueden hacer a la m\'aquina. La idea que ten\'ia Conway era encontrar un aut\'omata universal que no pareciera creado, que pareciera natural. La parte de la universalidad es lo que hace la busqueda complicada, en palabras simples, la universalidad significa que el aut\'omata puede hacer cualquier c\'alculo, algo as\'i como un computador que puede hacer multiplicaciones, sumas, divisiones, pero tambi\'en otras cosas, como guardar documentos, hojas de c\'alculo, reproducir videos. Los computadores y celulares que tenemos en nuestra vida diaria son aut\'omatas universales.

Conway empez\'o a buscar autómatas universales que tuvieran reglas simples. La idea de Von Neumann era sobre una cuadr\'icula de dos dimensiones, as\'i que Conway pens\'o primero en algo m\'as simple, algo de una sola dimensi\'on, FRACTRAN fue el primer resultado. Si bien FRACTRAN es un aut\'omata, FRACTRAN es mejor descrito como un lenguaje de programaci\'on, esto es, un sistema de notaci\'on para escribir programas de computaci\'on. En FRACTRAN los programas de computaci\'on se escriben como listas finitas de fracciones, y en estas se puede escribir cualquier tipo de programa, se pueden sumar, multiplicar, calcular los d\'igitos de $\pi$, etc. Si bien los programas en FRACTRAN son simples y FRACTRAN es universal, FRACTRAN se alejaba de la idea principal de su \textit{Jugendtraum}, ni tampoco era parecida a la idea de Von Neumann.

Despues de esto, Conway intent\'o jugar con aut\'omatas de una dimension, es decir aut\'omatas donde la cuadr\'icula es de solo un cuadro de ancho. Sin embargo, se le hizo muy dificil tener resultados sobre estos \cite{Roberts2015-ur}. Fue hasta que lo intent\'o con dos dimensiones cuando lleg\'o a conseguir su cometido.

\subsection{Life}

\label{subsection:life-explain}

Conway y aquellos colaboradores a los que tambi\'en les gustaba su idea trabajaron en varias versiones de aut\'omatas de dos dimensiones, durante mas o menos 18 meses. Quer\'ian reglas simples, enotnces empezaron a pensar en t\'erminos de vida y muerte, la idea era tener reglas para que una celda `viviera' y para que una celda `muriera', en ese sentido la celda ten\'ia dos estados. Algo importante era encontrar un buen balance entre las dos reglas, si las reglas sobre la muerte eran muy estrictas entonces todas las celdas terminar\'ian muertas pasados unas cuantas iteraciones, y si las reglas de la vida eran muy suaves entonces los patrones crecer\'ian much\'isimo sin dar tiempo a poder hacer algo \'util con ellos.

En la pen\'ultima versi\'on del juego de la vida, las celdas ten\'ian tres posibles estados, la idea era darles un `sexo' a las celdas vivas, dando los posibles estado `macho', `hembra' y `muerto'. En el blog de Tanya Khonakova \cite{blog:tanyakhovanova}, Conway cuenta que los nombres que les pusieron eran `obispos' y `actrices' seg\'un un chiste brit\'anico. El problema que ten\'ia esta idea era que para que las reglas estuvieran bien balanceadas, las celdas que nacieran necesitaban que hubiera tres celdas vivas adyacentes, en este caso tendr\'ia que haber dos `padres' con el mismo sexo y la nueva celda tendr\'ia que tener el sexo contrario para que hubiera un balance entre ambos sexos, era un tanto complicado y al final vieron que el sexo no ten\'ia m\'as efecto que determinar el sexo de la nueva celda, por lo tanto la solucion que encontraron fue hacer un aut\'omata asexual, solamente dos estados, `vivos' y `muertos', as\'i llegaron a la versi\'on final.

El juego de la vida se juega en un tablero dos dimensional conformado por celdas, como un tablero de Go pero infinito. El juego empieza con un estado inicial y va evolucionando conforme a las reglas. Cada celda tiene 8 celdas adyacentes a los que llamaremos vecinos, cuatro que comparten un lado y cuatro que comparten solamente un v\'ertice, es decir, dos horizontales, dos verticales, y cuatro en diagonal. Las reglas del juego son:

\begin{enumerate}
    \item \textbf{Regla de nacimiento:} Si una celda est\'a muerta en el tiempo $t$, y la celda tiene exactamente $3$ celdas vecinas que est\'an vivas, entonces la celda se convierte en una celda viva en el tiempo $t+1$.
    \item \textbf{Regla de muerte:} Si una celda viva en el tiempo $t$ tiene menos que $2$ vecinos vivos, la celda muere por soledad en el tiempo $t+1$. Si una celda viva en el tiempo $t$ tiene m\'as que $3$ vecinos vivos, la celda muere por hacinamiento en el tiempo $t+1$.
    \item \textbf{Regla de supervivencia:} Si una celda viva tiene exactamente $2$ o $3$ celdas vivas vecinas en el tiempo $t$, entonces la celda sigue viva en el tiempo $t+1$.
\end{enumerate}

Hagamos un ejemplo. Supongamos que de estado inicial tenemos tres celdas vivas puestas verticalmente, como en la figura \ref{figure:blinker}. Despu\'es del primer paso, la celda que est\'a en el medio ten\'ia dos celdas adyacentes vivas, entonces la celda sobrevivira, mientras que las celdas de arriba y abajo solo ten\'ian una celda vecina viva, por lo tanto estas celdas morir\'an en la siguiente iteraci\'on. Sin embargo, las celdas que se encuentran justo a los lados derecho e izquierdo de la celda del frente ten\'ian exactamente tres vecinos vivos, por lo tanto nacer\'an en el siguiente paso.

\begin{figure}[h]
    \centering
    \begin{subfigure}{0.4\textwidth}
        \includegraphics[width=\textwidth]{images/life-1.png}
    \end{subfigure}
    \begin{subfigure}{0.4\textwidth}
        \includegraphics[width=\textwidth]{images/life-2.png}
    \end{subfigure}
    \caption{En la izquierda se muestra la configuraci\'on en el paso $t=0$, a la derecha en la iteraci\'on siguiente $t=1$. A esta figura se le conoce como \textit{blinker}.}
    \label{figure:blinker}
\end{figure}


Los \textit{blinker} de la figura \ref{figure:blinker} tienen la propiedad de que se repiten, es decir, vuelven a su estado inicial. En el caso del \textit{blinker} se puede ver que por simetr\'ia, pasa de ser tres celdas verticales a ser tres celdas horizontales y viceversa. A los patrones que tienen la propiedad de volver a su estado inicial se les llama \textit{osciladores}, y a la cantidad de pasos que se demoran para volver a su estado inicial se le llama \textit{periodo}. Aunque hay much\'isimas figuras que son osciladores de periodo $2$, no todos los osciladores tienen periodo $2$, por ejemplo el \textit{pulsar} que fue tambi\'en descubierto por Conway (ver figura \ref{figure:pulsar}). Se han descubierto osciladores de much\'isimos periodos, tambi\'en se ha descubierto que se pueden construir osciladores de periodos mayores o iguales que $43$, sin embargo a\'un no se han descubierto osciladores con periodo $19$ o periodo $41$, siendo este a\'un un problema abierto.

\begin{figure}[h]
    \centering
    \begin{subfigure}{0.3\textwidth}
        \includegraphics[width=\textwidth]{images/life-pulsar-1.png}
    \end{subfigure}
    \begin{subfigure}{0.3\textwidth}
        \includegraphics[width=\textwidth]{images/life-pulsar-2.png}
    \end{subfigure}
    \begin{subfigure}{0.3\textwidth}
        \includegraphics[width=\textwidth]{images/life-pulsar-3.png}
    \end{subfigure}
    \caption{En la figura se pueden ver los pasos de la evoluci\'on de un patr\'on llamado \textit{pulsar}. Es un oscilador de periodo 3, por lo tanto despu\'es del tercer paso vuelve a su estado inicial.}
    \label{figure:pulsar}
\end{figure}

Adem\'as de los osciladores, tambi\'en exist\'ian otros patrones que no cambiaban en absoluto al pasar el tiempo, a estos patrones se les llama \textit{estacionarios}. El m\'as simple de estos patrones es el \textit{bloque}, conformado por cuatro celdas vivas unidas en forma de cuadrado, ninguna muere porque cada una est\'a en contacto con exactamente $3$ celdas vivas mientras que las celdas muertas alrededor solamente est\'an en contacto con $2$ celdas vivas, por lo tanto no pueden nacer. Se puede ver el bloque en la parte izquierda de la figura \ref{figure:stationary}.

\begin{figure}[h]
    \centering
    \includegraphics[width=0.7\textwidth]{images/life-stationary.png}
    \caption{Cinco patrones estacionarios, es decir, ninguno cambia con el tiempo. Sus nombres de izquierda a derecha: un \textit{bloque}, una \textit{colmena}, una \textit{tajada de pan}, un \textit{bote}, un \textit{balde}.}
    \label{figure:stationary}
\end{figure}

De las primeros patrones que se estudiaron, problablemente los m\'as interesantes son los llamados \textit{Matusal\'en}, nombrados en referencia al personaje m\'as longevo de la biblia. Estos patrones tienen en com\'un que evolucionan por largos periodos de tiempo antes de estabilizarse, es decir, en volverse figuras conocidas, como los osciladores o las figuras estacionarias. El primer Matusal\'en que se encontr\'o fue el R-pentomin\'o (ver figura \ref{figure:pentomino-r}), que tambi\'en fue una de las primeras evidencias de la complejidad que podr\'ia traer el juego de la vida. El R-pentomin\'o se demora 1103 iteraciones en estabilizarse, por lo tanto, fue un gran desaf\'io para los primeros investigadores del juego de la vida, que b\'asicamente hac\'ian toda su investigaci\'on en tableros de Go y a mano.

\begin{figure}[h]
    \centering
    \includegraphics[width=.4\textwidth]{images/life-r-pentomino.png}
    \caption{El R-pentomin\'o.}
    \label{figure:pentomino-r}
\end{figure}

De las figuras m\'as simples que se pueden formar con celdas son los poliomin\'os, un poliomin\'o es un objeto formado cuando se unen varias celdas cuadradas por sus lados, en particular, los pentomin\'os son poliomin\'os con cinco celdas, hay en total $12$ pentomin\'os diferentes y estos se nombran con las letras del alfabeto, de ah\'i sale el R-pentomin\'o. La complejidad del R-pentomin\'o se descubri\'o porque Conway y su equipo estaban probando todos los patrones simples que conocieran, empezaron con los poliomin\'o y se dieron cuenta que la mayor\'ia de ellos se estabilizaba apenas pasaban unas cuantas iteraciones, todos los dem\'as poliomin\'os con menos de 6 celdas vivas se estabilizan antes de la onceava iteraci\'on. 

Como Conway y su equipo hac\'ian el trabajo de revisar los patrones de manera an\'aloga con tableros de Go y a mano, se comte\'ian muchos errores. Al equipo de ellos lleg\'o a ayudarlos Richard Guy, un matem\'atico de Cambridge, que tuvo el trabajo de revisar y escrutinar la evoluci\'on de los patrones de los que a\'un no conoc\'ian todo, entre esos, el R-pentomin\'o. En la iteraci\'on n\'umero $69$, Richard Guy atisb\'o un nuevo patr\'on que se separaba del caos formado por el R-pentomin\'o y en las siguientes generaciones se apartaba m\'as y m\'as, cada vez como si caminara a trav\'es del tablero. Guy se lo mostr\'o a Conway, y este \'ultimo lo bautiz\'o \textit{glider}\footnote{Su traducci\'on al espa\~nol ser\'ia planeador, haciendo referencia a los aviones planeadores}.

\begin{figure}[b]
    \centering
    \begin{subfigure}{0.15\textwidth}
        \includegraphics[width=\textwidth]{images/life-glider-1.png}
    \end{subfigure}
    \begin{subfigure}{0.15\textwidth}
        \includegraphics[width=\textwidth]{images/life-glider-2.png}
    \end{subfigure}
    \begin{subfigure}{0.15\textwidth}
        \includegraphics[width=\textwidth]{images/life-glider-3.png}
    \end{subfigure}
    \begin{subfigure}{0.15\textwidth}
        \includegraphics[width=\textwidth]{images/life-glider-4.png}
    \end{subfigure}
    \begin{subfigure}{0.15\textwidth}
        \includegraphics[width=\textwidth]{images/life-glider-5.png}
    \end{subfigure}
    \caption{La evoluci\'on del \textit{glider}. Despu\'es de 4 iteraciones se transforma por una copia del mismo pero una casilla corrida a la derecha y abajo.}
\end{figure}

Este \textit{glider} era una se\~nal contundente de que el juego de la vida pod\'ia ser universal. Para que un aut\'omata se asemeje a un computador (en el hecho de ser universal) tiene que poder enviar informaci\'on de alguna manera, cuando Conway y su equipo encontraron el \textit{glider} se imaginaron que podr\'ian enviar informaci\'on mediante este patr\'on, an\'alogo a lo que se hace con la corriente el\'ectrica en la vida real. Para poder hacerlo se necesita una forma de generar \textit{gliders} peri\'odica y controladamente, algo as\'i como un ca\~non de gliders. En el art\'iculo de Gardner \cite{Gardner1970}, el reto de los 50 USD hac\'ia referencia a que un ca\~non de \textit{gliders} ser\'ia perfecto para ganar la apuesta, porque por un lado ser\'ia un patr\'on sin l\'imite de crecimiento entonces cumplir\'ia la condici\'on, y por otro les ayudar\'ia a los investigadores a probar la universalidad del juego de la vida.

En Noviembre del a\~no en que sali\'o la columna de Gardner (1970), un equipo del MIT liderado por Bill Gosper, un matem\'atico y programador estadounidense, se gan\'o el premio propuesto en la columna al descubrir/inventar una pistola de gliders ahora conocida como la \textit{Gosper Glider gun} (ver figura \ref{figure:gosper-gun}). 

\begin{figure}[h]
    \centering
    \includegraphics[width=.8\textwidth]{images/life-gosper-gun.png}
    \caption{La \textit{Gosper glider gun}. Genera un glider en la iteraci\'on 15, y a partir de esa genera un glider cada 30 iteraciones.}
    \label{figure:gosper-gun}
\end{figure}

Despu\'es de la carrera por encontrar una pistola de gliders, vino la carrera para mostrar que el juego de la vida era universal. Despu\'es del glider se econtraron muchas otras `naves espaciales', muchas m\'as grandes, y se estudi\'o tambi\'en c\'omo sus interacciones (o choques) podr\'ian ser \'utiles para mostrar la universalidad.

Para mostrar la universalidad toca mostrar que es posible formar cada una de las partes de un computador en el juego de la vida, por ejemplo, se tiene que mostrar que se puede modelar una memoria, o tambi\'en que se pueden modelar los c\'alculos que puede hacer un computador. Para los c\'alculos los computadores utilizan compuertas l\'ogicas, la \'idea es que por un lado de la compuerta entran una o dos se\~nales el\'ectricas, y dependiendo de estas por la otra sale o no sale corriente. Por ejemplo, la compuerta NOT lo que hace es invertir la se\~nal, si entra corriente a la compuerta entonces no sale corriente, y si no entra corriente entonces s\'i sale corriente.

En el juego de la vida, la forma de construir las compuertas l\'ogicas es mediante las propiedades de las colisiones de los gliders, hagamos el ejemplo de la compuerta NOT, en este caso tendremos una ``corriente'' de gliders tal que al llegar a la compuerta se les aplicar\'a el NOT, entonces si en ese momento llega un glider la compuerta no generar\'a ning\'un glider y si no llega ning\'un glider la compuerta s\'i generar\'a uno. Si dos gliders se chocan de una manera precisa, resultara en la destrucci\'on de los dos gliders, tal como se muestra en la figura \ref{figure:glider-collission}. Por lo tanto, lo que podemos hacer es chocar la corriente a la que le queremos aplicar la compuerta con una corriente de gliders constante, de modo que, si no se chocan la corriente constante seguir\'a produciendo gliders y si en cambio s\'i se chocan, todos los producidos por la corriente se destruir\'an.

\begin{figure}[h]
    \centering
    \begin{subfigure}{0.22\textwidth}
        \includegraphics[width=\textwidth]{images/life-glider-collission-1.png}
    \end{subfigure}
    \begin{subfigure}{0.22\textwidth}
        \includegraphics[width=\textwidth]{images/life-glider-collission-2.png}
    \end{subfigure}
    \begin{subfigure}{0.22\textwidth}
        \includegraphics[width=\textwidth]{images/life-glider-collission-3.png}
    \end{subfigure}
    \begin{subfigure}{0.22\textwidth}
        \includegraphics[width=\textwidth]{images/life-glider-collission-4.png}
    \end{subfigure}
    \caption{Dos gliders que van en direcciones opuestas chocan destruyendose mutuamente. Fotos tomadas en el tiempo t=0,2,4,6, de izquierda a derecha.}
    \label{figure:glider-collission}
\end{figure}

Si bien la existencia de compuertas l\'ogicas de por s\'i no implica que el juego de la vida fuera universal, se fueron descubriendo varios patrones que implicaban de cierta manera que se pod\'ia construir las diferentes partes de un computador en el juego de la vida; se descubrieron formas de guardar corrientes de gliders, es decir memoria, y se descubrieron las otras compuertas.

Un mes despu\'es de que Bill Gosper descubriera la pistola de gliders que lleva su nombre, Conway le escribi\'o una carta a Martin Gardner que dec\'ia \cite{Roberts2015-ur}:

\begin{samepage}
    ``\textit{Querido MG,} 

    \textit { Espero que esta carta no te llegue muy tarde - el correo parece que se demora mucho en llegarte. Retras\'e esta carta hasta ahora porque quer\'ia completar mi prueba de que EL JUEGO DE LA VIDA ES UNIVERSAL.}''\footnote{Traducci\'on del autor.}
\end{samepage}

Si bien Conway nunca public\'o su prueba de que el juego de la vida es universal, el juego se volvi\'o tan famoso que hasta la fecha se han hecho un mont\'on de construcciones en el juego de la vida que lo prueban. En ese entonces se estimaba que para hacer un computador en el juego de la vida se necesitar\'ia millones de celdas, sin embargo, ahora se sabe, gracias a construcciones hechas por comunidades en l\'inea, que no se necesita tantas celdas para hacer un computador, e incluso se han creado computadores que tienen sus propias pantallas en el mismo juego \cite{YTComputerGL}.