\section{Juegos Matem\'aticos}

A Conway siempre le interesaron los juegos, era fan\'atico al backgammon y de sus primeros aportes a la columna de Martin Gardner eran juegos que \'el mismo hab\'ia creado, por ejemplo el juego Sprouts, creado por Conway en compa\~n\'ia del entonces estudiante de posgrado Mike Paterson, fue descrito en una carta que Conway le envi\'o a Gardner en 1967 y ese mismo a\~no Gardner la public\'o en su columna \textit{Mathematical Games}\cite{Gardner2005-vo}.

Por eso no es sorpresa, que alrededor del a\~no 1970, al descubrir el teorema de Sprague-Grundy, se obsesionara con los juegos. El teorema de Sprague-Grundy\cite{Grundy1939}, es un teorema que trata sobre los juegos imparciales, estos son aquellos juegos de informaci\'on perfecta, sin azar, tales que los movimientos posibles dependan solamente del tablero y no de qu\'e jugador est\'e jugando.

Vamos a explicar esta definici\'on parte por parte. Un juego de informaci\'on perfecta es un juego donde cada persona sabe cu\'ales movimientos pueden hacer los otros jugadores, en este sentido juegos como el p\'oker o el black-jack quedan descartados, puesto que los jugadores no saben qu\'e cartas tienen los otros. Un juego sin azar es un juego donde los movimientos no dependen del azar, no hay ning\'un dado que dicte los posibles movimientos ni hay ning\'un generador de n\'umeros aleatorios que dicte el ganador, en este caso quedan descartados juegos como el parqu\'es, parch\'is o la loter\'ia. La \'ultima condici\'on de estos juegos es la que le da el nombre de imparciales, esto es, si tenemos la posici\'on del tablero los movimientos posibles para los jugadores son los mismos, aqu\'i se descartan la mayor\'ia de otros juegos que se conocen, por ejemplo, el ajedrez tiene diferentes movimientos para los dos jugadores, uno juega con las fichas blancas y el otro con las fichas negras, o por ejemplo el triqui (tic-tac-toe), en este un jugador juega con las cruces (`X') y otro con los c\'irculos (`O'). F\'ijese en las figuras \ref{figure:game-chess} y \ref{figure:game-triqui}, en ambos tableros los posibles movimientos para las fichas negras y para las fichas blancas son diferentes, en ambos casos el jugador de las fichas negras puede ganar en el siguiente movimiento mientras que el de las fichas blancas no lo puede hacer.

\TwoFig{images/games-partisan-1.png} % image 1
         {Tablero de ajedrez donde el jugador con las fichas negras puede ganar en un movimiento, pero el jugador con las blancas no. \textit{(Fuente: lichess.com)}} % caption 1
         {figure:game-chess} % label 1
         {images/games-partisan-2.png} % image 2
         {Tablero de triqui donde el jugador con las `X' puede ganar en un movimiento, pero el jugador con las `O' no. \textit{(Google)}} % caption 2
         {figure:game-triqui} % label 2

Un ejemplo de juego al que s\'i se le puede aplicar el teorema de Sprague-Grundy, es decir un juego imparcial, es el comunmente llamado juego 21. En el juego se van turnando entre dos personas, el primer jugador dice el n\'umero `1' y el siguiente jugador tiene que aumentar el n\'umero en `1', `2' o `3'. Pierde el jugador que diga `21' o un n\'umero mayor que `21'. Un ejemplo de un juego de `21' es:

\begin{center}
    \begin{tabular}{|c|M{3cm}|}
        \hline
        Jugador & \text{N\'umero} \\
        \hline\hline
        1 & 1 \\
        \hline
        2 & 4 \\
        \hline
        1 & 7 \\
        \hline
        2 & 10 \\
        \hline
        1 & 11 \\
        \hline
        2 & 13 \\
        \hline
        1 & 16 \\
        \hline
        2 & 17 \\
        \hline
        1 & 20 \\
        \hline
        2 & 21 \\
        \hline
    \end{tabular}
\end{center}

Ac\'a el jugador 1 gan\'o porque el jugador 2 dijo `21'. F\'ijese que en este juego las opciones para los dos jugadores son las mismas, el `tablero' de este juego consistir\'ia en el n\'umero que se ha dicho inmediatamente en el turno anterior.

El teorema de Sprague-Grundy\footnote{Formalmente, el teorema dice que todo juego imparcial es equivalente a un `nimber', el an\'alogo en juegos imparciales a los n\'umeros que luego definiremos.} le asigna un valor a todos los juegos de este tipo con el que se puede calcular qu\'e jugador va a ganar si ambos jugadores juegan de la mejor manera. Adem\'as de todo, el teorema da una forma en la que se pueden calcular valores de juegos que consisten en la combinaci\'on de otros juegos m\'as simples, por ejemplo, si quisieramos jugar dos juegos imparciales al mismo tiempo pero en cada turno solo podemos hacer un movimiento en uno de los dos juegos, el teorema de Sprague-Grundy nos dir\'ia como calcular qui\'en ganar\'ia con base a los valores de cada uno de los juegos imparciales.

Conway quer\'ia generalizar este teorema para otra clase de juegos m\'as generales, los llamados juegos partisanos. Los juegos partisanos son tambi\'en juegos sin azar y de informaci\'on perfecta, pero en este caso no tenemos la condici\'on extra de imparcialidad, entonces entran juegos como el ajedrez, el triqui o el Go. Todos los juegos imparciales son a su vez juegos partisanos.

En la \'epoca de 1970 sucedieron var\'ias cosas que ayudaron al estudio de estos juegos. Primero, Jon Diamond, el campe\'on brit\'anico de Go durante 11 a\~nos \cite{britgo_JON_DIAMOND}, entr\'o a estudiar matem\'aticas en Cambridge. \'El fund\'o la \textit{Cambdridge Go Society} y con eso incentiv\'o el juego del Go dentro de Cambridge. Conway se interes\'o por el Go principalmente porque este juego tiene la propiedad de que casi al final de la partida, el tablero se separa en ciertas zonas, como si se estuvieran jugando varios juegos simult\'aneamente, cada jugador eligiendo en cada turno en qu\'e parte jugar.

La otra cosa que ayud\'o al desarrollo de la teor\'ia fue el inter\'es de Richard Guy y Elwyn Berlekamp en los juegos. Elwyn Berlekamp fue un matem\'atico y cient\'ifico de la computaci\'on que ya hab\'ia trabajado en juegos antes, su tesis de maestr\'ia hab\'ia sido en algoritmos para resolver problemas del bridge (el juego de cartas), y mientras trabajaba en Bell Labs hizo un reporte t\'ecino del juego de cuadritos, un juego a l\'apiz y papel donde los jugadores se turnan conectando puntos adyacentes en una cuadr\'icula y gana el que haya hecho m\'as cuadritos. Richard Guy, el mismo que descubri\'o el \textit{glider} del juego de la vida, conoc\'ia a los dos y fue el puente para que los tres trabajaran juntos.

Los tres dieron vida a \textit{Winning Ways for Your Mathematical Plays}, un compendio de cuatro vol\'umenes con toda la informaci\'on que ten\'ian sobre juegos matem\'aticos. Un trabajo que se demor\'o quince a\~nos en salir a la luz, con dos vol\'umenes, y luego en una segunda edici\'on 20 a\~nos despu\'es con dos vol\'umenes m\'as.

En este compendi\'o desarrollaron la teor\'ia de los juegos partisanos y la extendieron. En los libros aplican la teor\'ia a much\'isimos juegos diversos, tanto a juegos imparciales como a juegos partisanos, y extienden el teorema de Sprague-Grundy a juegos partisanos tambi\'en.

\subsection{Hackenbush}

Un juego que nos permite explicar una parte de la teor\'ia de manera m\'as directa es el juego del Hackenbush. El juego lo invent\'o Conway pero el nombre le fue dado por Richard Guy inspirado en un personaje de la pel\'icula de 1937 \textit{A Day at the Races}, le puso as\'i porque el juego consiste en cortar (\textit{hack} en ingl\'es) arbustos (\textit{bushes} en ingl\'es).

\begin{figure}[h]
    \centering
    \includegraphics[width=.7\textwidth]{images/hackenbush-first_example.pdf}
    \caption{Una posible configuraci\'on inicial del juego del Hackenbush.}
\end{figure}

El juego empieza con una configuraci\'on de l\'ineas, unas rojas y otras azules. Todas las l\'ineas tienen que estar conectadas al piso, ya sea directamente o a trav\'es de otras. El juego se juega por turnos, cada uno de los jugadores tiene un color asignado, un jugador juega con las azules y el otro juega con las rojas. En cada turno el jugador tiene que borrar una l\'inea de su color, y si alguna de las otras l\'ineas queda desconectada entonces esa tambi\'en se borra. Es como si se estuvieran cortando arbustos, si se corta el tallo del arbusto entonces este se va a caer al piso y ya no va a estar en el juego.


\begin{figure}[h]
    \centering
    \includegraphics[width=.7\textwidth]{images/hackenbush-game_example.pdf}
    \caption{Jugando Hackenbush.}
\end{figure}


Lo interesante del juego es que tiene la propiedad que estaba buscando Conway en juegos partisanos: algunas configuraciones tienen ciertos `arbustos' desconectados unos de otros, por lo tanto cada uno de estos arbustos se puede tratar como un juego separado, as\'i como lo muestra la imagen de la figura \ref{figure:hackenbush_sum}. Vamos a usar el hackenbush en lo que queda de esta secci\'on para mostrar c\'omo se calculan los valores de un juego en general. Para todos los juegos se puede hacer una construcci\'on parecida pero puede que aplicar la teor\'ia sea m\'as dificil. Lo que hace excepcional al hackenbush, es que parece un juego diseñado para que ilustrar la teor\'ia sea m\'as sencillo.

\begin{figure}[h]
    \centering
    \includegraphics[width=.7\textwidth]{images/hackenbush-sum_example.pdf}
    \caption{La puerta y el contorno de la casa se pueden tomar como juegos separados.}
    \label{figure:hackenbush_sum}
\end{figure}


\subsection{Valores de juegos}

\label{subsection:game_values}

Para un estado del juego, en este caso el hackenbush, tenemos los posibles movimientos que puede hacer el jugador de las l\'ineas azules y los posibles movimientos que puede hacer el jugador de las l\'ineas rojas. Si calculamos los valores para esos estados antes vamos a poder calcular el valor de nuestro estado con respecto a esos posibles estados siguientes.

Sea $L$ el conjunto de los valores de los estados a los que puede llegar el jugador azul, y sea $R$ el del jugador rojo, entonces el valor de nuestro estado se va a definir como $\surr{L}{R}$. En la figura \ref{figure:hackenbush_val} podemos ver un ejemplo de esto, si bien en la figura se muestra los estados siguientes la idea es guardar solamente los valores.

\begin{figure}[h]
    \centering
    \includegraphics[width=.7\textwidth]{images/hackenbush-val_example.pdf}
    \caption{A la izquierda el estado del juego, y a la derecha sus movimientos. Entre los par\'entesis a la izquierda se ve el siguiente estado si el jugador azul juega, y a la derecha el del jugador rojo.}
    \label{figure:hackenbush_val}
\end{figure}

En cierto sentido, los valores est\'an definidos a partir de los valores siguientes, y estos valores siguientes est\'an definidos a partir de los siguientes, y as\'i hasta que se acabe el juego. El hackenbush termina cuando no hay ninguna l\'inea en el estado, por lo tanto vamos a tomar este como nuestro caso base, a ese estado le asignaremos un valor de $0$. 

En general cuando analizamos un juego es mejor empezar desde sus \'ultimos movimientos, sus movimientos m\'as b\'asicos, para ver c\'omo se puede ganar. En el caso del hackenbush los siguientes movimientos m\'as b\'asicos son cuando existe solo una l\'inea, sea roja o azul. Si la l\'inea es azul entonces el valor es $\surr{0}{}$, mientras que si la l\'inea es roja el valor es $\surr{}{0}$, a estos valores los llamaremos $1$ y $-1$ respectivamente, as\'i como lo indica la figura \ref{figure:hackenbush_roots}.

\begin{figure}[h]
    \centering
    \includegraphics[width=.7\textwidth]{images/hackenbush-surr_roots.pdf}
    \caption{Diagrama de los casos base. Cada flecha indica un movimiento, y el color corresponde al jugador que lo hace.}
    \label{figure:hackenbush_roots}
\end{figure}

Lo que intentan capturar estos valores es el sesgo que tiene el tablero para un jugador o el otro, los tableros con valores positivos van a ser ganados por el jugador azul y los de valores negativos van a ser ganados por el jugador rojo. Pero hay otro tipo de tableros, aquellos que no tienen ning\'un sesgo de color, como el de la derecha de la figura \ref{figure:hackenbush_roots}. En este tablero gana el jugador que juegue en segundo lugar, esto es, si empieza el azul gana el rojo, y si empieza el rojo gana el azul, en estos tableros el valor va a ser igual a $0$, por lo tanto tenemos que $0=\surr{-1}{1}$.

Esta definici\'on de $0=\surr{-1}{1}$ es consistente con lo que hab\'iamos dicho de sumar los juegos. Otra forma de verificar el valor de este juego es darse cuenta que este juego equivale a la suma de dos juegos, el de la izquierda es solo una l\'inea azul, y el de la derecha es solo una l\'inea roja, estos juegos valen respectivamente $1$ y $-1$, por lo tanto el valor del juego va a ser igual al valor de su suma, $1+(-1) = 0$.

En la secci\'on \ref{section:surreal} de n\'umeros surreales nos acercamos a esta definici\'on desde otro \'ambito.

\subsection{Comparando juegos}

Ya teniendo un juego de valor $1$, y otro juego de valor $-1$, podemos crear juegos de valores $n$ y de valores $-n$ para todo $n$ n\'umero natural. Estos ser\'ian $n$ l\'ineas de color azul para el caso de un juego de valor $n$, y $n$ l\'ineas de color rojo para un juego de valor $-n$. Con estos ya podemos intentar calcular los valores de otros juegos m\'as complejos.

Un ejemplo es calcular el valor de varias l\'ineas del mismo color unidas verticalmente. Consideremos la figura \ref{figure:hackenbush-integer_val}, a la izquierda se puede ver tres l\'ineas rojas unidas verticalmente, y a la derecha tres l\'ineas azules separadas. El valor de la parte derecha, la parte azul, se puede calcular ya que es la suma de $3$ l\'ineas azules simples, es decir, el valor es $3$.

\begin{figure}[h]
    \centering
    \includegraphics[width=.5\textwidth]{images/hackenbush-integer_val.pdf}
    \caption{En este juego siempre gana el segundo jugador.}
    \label{figure:hackenbush-integer_val}
\end{figure}

Ahora, si analizamos el juego podemos ver que, si los dos jugadores juegan \'optimamente, el segundo jugador siempre ganar\'a sin importar que color empiece: El jugador de color rojo va a quitar la l\'inea m\'as arriba y el jugador de color azul va a quitar una l\'inea azul, en cada turno va a haber siempre una l\'inea menos de ambos colores, luego gana el que juegue de segundas. Esto significa que el juego tiene valor $0$, entonces como la parte derecha azul tiene valor $3$ la parte izquierda tiene que tener valor $-3$.

Para todos los juegos hechos a partir de l\'ineas del mismo color unidas verticalmente tenemos que el valor es la cantidad de l\'ineas unidas, $n$ si es azul y $-n$ si es rojo, siendo $n$ la cantidad de l\'ineas. 

Si nos fijamos en la definici\'on de los n\'umeros de estos juegos podemos darnos cuenta de otra igualdad. Para un juego conformado por $n$ l\'ineas azules simples separadas, nuestro valor ser\'a $\surr{n-1}{}$, puesto que sin importar que l\'inea quitemos siempre quedar\'an $n-1$ l\'ineas. Para un juego de $n$ l\'ineas azules est\'an unidas verticalmente tendremos que el valor ser\'a $\surr{0,1,2,\dots,n-1}{}$, puesto que en estos estados podemos quitar la cantidad de l\'ineas que querramos en un solo movimiento.
Como ambos juegos tienen el mismo valor, entonces lo que acabamos de probar nos da la igualdad $\surr{n-1}{} = \surr{0,1,2,\dots,n-1}{}$, que es una igualdad abstracta que pasa en cualquier juego, no solo en Hackenbush.

Hagamos otro ejemplo, consideremos dos l\'ineas pegadas verticalmente, la de abajo azul y la de arriba rojo, exactamente como en la figura \ref{figure:hackenbush-half}. Si vamos por todos los posibles movimientos de los dos jugadores nos damos cuenta que el valor es positivo, ya que el azul siempre va a ganar, pero, ¿Exactamente cu\'al es el valor?

\begin{figure}[h]
    \centering
    \includegraphics[width=.5\textwidth]{images/hackenbush-half_def.pdf}
    \caption{El valor del diagrama calculado.}
    \label{figure:hackenbush-half}
\end{figure}

Algo que podemos hacer es compararlo con juegos de valor $-n$, si al sumarlo con el juego todav\'ia sigue ganando el azul esto significa que el valor del juego es menor que $n$. Haciendo esto podemos ver que el valor de nuestro juego est\'a entre $0$ y $1$, esto es, si le sumamos un juego de una l\'inea roja al nuestro entonces gana el rojo. En otras palabras, $0 < \surr{0}{1} < 1$. Lo interesante de $\surr{0}{1}$ es que entonces no es un valor entero, por lo tanto lo siguiente que tendremos que hacer es ver entre qu\'e valores racionales est\'a.

Para esto, lo primero que hacemos es sumar dos veces este juego y compararlo con un juego de valor $-1$, es decir, con un juego de una l\'inea roja, nuestro juego quedar\'ia como en la parte izquierda de la figura \ref{figure:hackenbush-half_proof}. Si en este juego gana el jugador rojo eso significa que dos veces el valor del juego es menor que $1$, si gana el jugador azul significa que es mayor que $1$, y si gana el segundo jugador esto significa que el valor es precisamente $\frac{1}{2}$ ya que dos veces el juego es exactamente $1$.

\begin{figure}[h]
    \centering
    \includegraphics[width=.7\textwidth]{images/hackenbush-half_proof.pdf}
    \caption{Demostraci\'on gr\'afica de que $\surr{0}{1}=\frac{1}{2}$.}
    \label{figure:hackenbush-half_proof}
\end{figure}

Lo interesante de este an\'alisis es que, con argumentos parecidos podemos calcular el valor de cualquier juego de Hackenbush, y, sabiendo propiedades de estos valores, como las demostradas en la secci\'on \ref{section:surreal}, podemos hacer este c\'alculo incluso m\'as r\'apido.

Por ejemplo, una propiedad es que si el valor es $x=\surr{L}{R}$, entonces se puede mostrar que $x$ es mayor que todos los elementos de $L$ y menor que todos los elementos de $R$. Por lo tanto, si un elemento de $L$ es positivo o $0$, entonces $x$ es positivo, por lo tanto gana el azul. Esto se traducir\'ia a los juegos como, si el jugador azul puede llevar el juego a un estado donde sabemos que o gana el azul o gana el segundo jugador, entonces en ese estado gana el azul.

Estos valores, llamados n\'umeros surreales, se alimentan mutuamente del lenguaje de los juegos, esto es, propiedades de los juegos muestran propiedades en los n\'umeros surreales y propiedades en los n\'umeros pueden a su vez mostrar propiedades en los juegos.

Una muestra de esto son los inversos aditivos, si en un juego de hackenbush cambiamos el color de todas las l\'ineas entonces el valor de este nuevo juego ser\'a el inverso del valor del juego anterior. Esto lo podemos ver sumando los dos juegos: Si el jugador uno hace un movimiento, entonces el jugador dos puede jugar el mismo movimiento pero en la otra figura, asegurandose por simetr\'ia que el jugador dos siempre tendr\'a un movimiento por hacer, luego el valor de la suma es $0$ y los valores de las dos figuras son inversas aditivas.

\begin{figure}[h]
    \centering
    \includegraphics[width=.7\textwidth]{images/hackenbush-additive_inverse.pdf}
    \caption{La suma de cualquiera dos juegos con los valores cambiados da valor $0$.}
    \label{figure:hackenbush-add_inv}
\end{figure}

Si nos ponemos a calcular por definici\'on el valor de este inverso podemos ver que nos da la misma definici\'on que tenemos en nuestra seccion \ref{section:surreal}. Ac\'a los posibles movimientos del jugador rojo van a ser los mismos que ten\'ia antes el jugador azul y viceversa, por lo tanto se puede decir que si el valor de un juego es $\surr{L}{R}$ entonces el valor del inverso aditivo es $\surr{-R}{-L}$.
