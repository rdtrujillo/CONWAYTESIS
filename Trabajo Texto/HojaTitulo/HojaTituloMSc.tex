%\newpage
%\setcounter{page}{1}
\begin{center}
\begin{figure}
\centering%
\epsfig{file=HojaTitulo/EscudoUN.eps,scale=1}%
\end{figure}
\thispagestyle{empty} \vspace*{2.0cm} \textbf{\huge
Una antolog\'ia de la obra de Conway}\\[6.0cm]
\Large\textbf{Roger David Trujillo Ibáñez}\\[6.0cm]
\small Universidad Nacional de Colombia\\
Facultad de Ciencias, Departamento de Matemáticas\\
Bogotá, Colombia\\
2023\\
\end{center}

\newpage{\pagestyle{empty}\cleardoublepage}

\newpage
\begin{center}
\thispagestyle{empty} \vspace*{0cm} \textbf{\huge
Una antolog\'ia de la obra de Conway}\\[3.0cm]
\Large\textbf{Roger David Trujillo Ibáñez}\\[3.0cm]
\small Tesis o trabajo de grado presentada(o) como requisito parcial para optar al
t\'{\i}tulo de:\\
\textbf{Pregrado en Matemáticas}\\[2.5cm]
Directora:\\
Ph.D. Jeanneth Galeano Pe\~naloza\\[5.0cm]
Universidad Nacional de Colombia\\
Facultad de Ciencias, Departamento de Matemáticas\\
Bogot\'a, Colombia\\
2023\\
\end{center}

\newpage{\pagestyle{empty}\cleardoublepage}

\newpage
\thispagestyle{empty} \textbf{}\normalsize
\\\\\\%
\\[4.0cm]

\begin{flushright}
\begin{minipage}{8cm}
    \noindent
        \small
        I sort of made a vow to myself, I was going to study whatever I thought was interesting and not worry whether this was serious enough\\\\
        John Conway\\
\end{minipage}
\end{flushright}

\newpage{\pagestyle{empty}\cleardoublepage}

\newpage
\thispagestyle{empty} \textbf{}\normalsize
\\\\\\%
\textbf{\LARGE Agradecimientos}
\addcontentsline{toc}{chapter}{Agradecimientos}\\\\
Tengo mucho que agradecer: Primero, agradezco a mi directora de tesis, la profesora Jeanneth Galeano, por su gu\'ia, por sus correcciones, y sobre todo por su paciencia. Este viaje empez\'o con muchas ideas de parte m\'ia y la profesora tuvo la paciencia de acompa\~narme para que se pudieran materializar un par de ellas. 

Segundo, quiero agradecer a la educaci\'on p\'ublica, de calidad, que me mostr\'o el camino durante once a\~nos hasta llegar hasta ac\'a, en especial a la Universidad Nacional donde aprend\'i a ser lo que soy en este momento.

Tercero, quiero agradecer a todas las personas que hicieron parte de este camino: a Jos\'e Alfredo, que me di\'o la idea de este trabajo y que sin \'el ya me hubiera salido de la carrera, a Mar\'ia Paula por acompa\~narme y escucharme todos estos meses, a Lauren Sof\'ia, que me acompa\~no en las noches en las que escrib\'ia estas p\'aginas y que me escuch\'o practicando la presentaci\'on del trabajo. A Jaime Zamora y a Juan Lara, que hicieron muchas materias de la carrera mucho m\'as sencillas. A Valeria, Daniel, Mateo, Aura, Mar\'ia, Juanfe, Isabela, Diego, Pronoia, y Pulldogs, por darme comunidad estos \'ultimos semestres de la carrrera. Much\'isimas gracias a mis compañeros de matemáticas que siguen siendo fuente de inspiración y motivación, a María Camila (gracias por todo), Daniel Checa, Emmanuel, Heldert, María Alejandra, Lina, Isaac, Jhonatan, Laura, Paulina, Torro, lo logramos. Muchas gracias a mis compañeros de Olimpiadas de computación, a Andrés, a Jhonny, a Osman y a Óscar, por todo lo bueno que me ha llegado gracias a eso. También gracias al profesor John Jaime que hizo esta colaboración posible. Pero sobre todo gracias a mis padres y mi abuelita, que me han acompañado siempre.

Gracias a Quantil por darme trabajo, a Olimpiadas de Computaci\'on y de Matem\'aticas por mostrarme el inicio del camino, al rap por darme las palabras, a mi gata Regina que siempre me acompaña y al software libre que dio vida a las gráficas de este proyecto (Inkscape).

Por \'ultimo, quiero agradecer a Owen Maitzen por haber hecho uno de los videos educativos con mejor producci\'on en YouTube, y que casualmente, trataba de teor\'ia de juegos combinatorios, su v\'ideo sobre el \textit{Hackenbush} \cite{YTHackenbush} fue una fuerte inspiraci\'on y motivaci\'on para seguir estudiando esto, y adem\'as, sus composiciones de piano me acompa\~naron en varios momentos mientras escrib\'ia este trabajo, que en paz descanse. 

Muchas gracias a Conway.

\newpage{\pagestyle{empty}\cleardoublepage}

\newpage
\textbf{\LARGE Resumen}
\addcontentsline{toc}{chapter}{Resumen}\\\\
Hacemos un recorrido divulgativo sobre algunos de los trabajos del matemático John Conway, tratando la historia del concepto y explicandolo matemáticamente. Tratamos diulgativamente el juego de la vida y sus aportes a la teoría de juegos combinatorios, también, hacemos una descripción formal de los números surreales, probamos que son un cuerpo ordenado y mostramos explícitamente la construcción de algunos de estos números.\\[2.0cm]

\textbf{\small Palabras clave: John Conway, Historia de las matemáticas, Números surreales, Teoría de juegos combinatorios, Juego de la vida de Conway.}.\\

\textbf{\LARGE Abstract}\\\\
We make a divulgative tour on some of the works of the mathematician John Conway, treating the history of the concept and explaining it mathematically. We treat diulgatively the game of life and its contributions to combinatorial game theory, also, we make a formal description of surreal numbers, prove that they are an ordered field and show explicitly the construction of some of these numbers.\\[2.0cm]
\textbf{\small Keywords: John Conway, History of mathematics, Surreal numbers, Combinatorial game theory, Conway's Game of Life.}\\