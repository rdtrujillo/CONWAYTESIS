\chapter{Introducci\'{o}n}

¿Qué hace que algunas ramas de las matemáticas tengan más atención que otras? ¿Qué hace que unas matemáticas sean más serias y tengan más premios y reconocimiento que otras que son igual o más dificiles? Las matemáticas están, como todas las actividades humanas, fuertemente ligadas a la cultura y al quehacer humano. Los diferentes rumbos que toma la matemática a través del tiempo, los diferentes premios que se otorgan, los diferentes incentivos para que la matemática sea de cierta forma, todo esto son solo ejemplos de la influencia que tenemos nosotros los seres humanos y nuestra cultura en las matemáticas. Este trabajo tiene el objetivo de resaltar, de cierta manera, esta presencia de la humanidad en las matemáticas.

John Conway fue un matemático poco usual, no trabajó en las ramas más populares, le gustaban mucho las matemáticas recreativas, y por esto mismo es bastante conocido por personas que no se dedican profesionalmente a las matemáticas. A lo largo de la investigación para este documento pude presenciar el profundo aporte de la obra de Conway a la matemática, un aporte que va mucho más allá de teoremas y resultados, un aporte que se mide mucho mejor en la cantidad de vidas matemáticas que fueron motivadas por su obra. Cuando se suele hablar de una persona que hace matemáticas casi siempre se omite esto último; a cuántas personas ayudaron sus clases, a cuántas personas apoyó en el camino hacia las matemáticas, cuántas personas se sintieron apoyadas por esta persona. Para m\'i, la parte más importante de la matemática est\'a en las mismas personas que conforman esta comunidad, las personas que se inspiran en esta comunidad, las que viven de esto, las que sueñan con esto.

Este trabajo quiere hacer énfasis en esta parte, la primera parte del trabajo es una recopilación divulgativa de dos trabajos matemáticos de John Conway. El primer trabajo, y también el más famoso de John Conway, es el juego de la vida, un autómata celular universal de dos dimensiones que inspiró a muchas comunidades de interesados en la informática y la matemática a jugarlo y teorizar sobre este; contamos como surgi\'o el concepto y todo lo que tuvo que pasar para poder demostrar al fin que el aut\'omata era universal, tambi\'en hacemos un peque\~no esbozo de su teor\'ia. El segundo trabajo es sobre juegos combinatorios, hacemos un recuento histórico de cómo surgió el trabajo entre él y sus colegas para crear \textit{Winning Ways for Your Mathematical Plays}, un recopilatorio grand\'isimo de todo lo que se sab\'ia sobre juegos combinatorios, y al final hacemos una explicaci\'on divulgativa de c\'omo se crean los valores de estos juegos y la forma en la que se estudian.

En la segunda parte, se hace una exposici\'on matem\'atica de los n\'umeros surreales, m\'as espec\'ificamente se demuestra que estos son un cuerpo ordenado y se hacen las construcciones de algunos n\'umeros expl\'icitamente. Conway dec\'ia que los n\'umeros surreales eran el trabajo matem\'atico que m\'as orgullo le daba. Para m\'i, los n\'umeros surreales son una demostraci\'on del poder de la curiosidad y la autenticidad en la matem\'atica, estos n\'umeros surgen del estudio de los juegos combinatorios y terminan siendo la extensi\'on m\'as grande de los n\'umeros reales, es decir, contienen a cualquier otro cuerpo ordenado.

Por \'ultimo, quiero hacer \'enfasis que si bien Conway trabaj\'o en much\'isimas otras ramas de las matem\'atica, quise enfocarme en el trabajo recreativo. Para m\'i, la matem\'atica recreativa tiene el potencial de ser extremadamente bella y tambi\'en extremadamente cercana a las personas en el sentido en que el objetivo es que cualquier persona la pueda entender. Este trabajo de grado es un intento de eso, de hacer la matem\'atica m\'as cercana. Si bien hay partes del texto donde los prerrequisitos para entenderlo sean un poco m\'as grandes que los que tenga una persona normal, intent\'e que para cualquier persona hubiera algo interesante en el libro. En ese sentido, s\'i es un intento de hacer la matem\'atica m\'as cercana, pero es un intento bien personal, un intento de hacer la matem\'atica m\'as cercana para m\'i, y que ojal\'a, en el proceso, tambi\'en pueda hacer la matem\'atica m\'as cercana para alguna otra persona.