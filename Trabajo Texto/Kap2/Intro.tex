John Conway fue un matem\'atico bastante prol\'ifico, seg\'un su p\'agina de Wikipedia \cite{wiki:John_Horton_Conway} Conway trabaj\'o en `teor\'ia de grupos finitos, teor\'ia de nudos, teor\'ia de n\'umeros, teor\'ia combinatoria de juegos, teor\'ia de c\'odigos y matem\'aticas recreacionales', en el libro \textit{The Mathematical Artist} \cite{MathematicalArtist2022} lo describen como un `\textit{polymath}' que transform\'o las matem\'aticas.

Sin embargo, exist\'ia un descubrimiento que lo enorgullec\'ia m\'as que el resto de sus aportes a las matem\'aticas, los n\'umeros surreales \cite{YTNumberphileConway}. Los n\'umeros surreales son una extensi\'on de los n\'umeros reales que admite infinitos e infinitesimales, adem\'as son el cuerpo ordenado m\'as grande, esto significa que todos los dem\'as cuerpos ordenados son subcuerpos de los n\'umeros surreales \cite{Bajnok2013}.

La primera descripci\'on formal de los n\'umeros surreales apareci\'o en el libro \textit{On numbers and Games} de Conway en 1976. Mientras estaba trabajando en \textit{Winning Ways for your Mathematical Plays} a Conway se le ocurri\'o la idea de escribir \textit{On Numbers and Games}. A Conway le preocupaba la idea de que hab\'ia una parte de la teor\'ia desarrollada en \textit{Winning Ways} que no iba a poder ser parte de ese libro, principalmente porque era una parte ``transfinita'' de la teor\'ia, entonces no se le ve\'ia a esta parte una aplicaci\'on concreta en los juegos \cite{Conway2000}. Entonces Conway se sent\'o por una semana a reunir todo el material que luego iba a conformar \textit{On Numbers and Games}, pero sin contarle a los dem\'as autores de \textit{Winning Ways} Elwyn Berlekamp y Richard Guy. Despu\'es de la semana de recolecci\'on, Conway les confes\'o a los coautores su intenci\'on de publicar \textit{On Numbers and Games} y Elwyn Berlekamp lo amenaz\'o con tomar acciones legales contra la publicaci\'on. Afortunadamente todo se resolvi\'o y se pudieron publicar estos dos trabajos, cada uno enfocados en cosas distintas; \textit{On numbers and Games} desarrolla la teor\'ia de los n\'umeros surreales como extensi\'on de los n\'umeros reales, mientras que \textit{Winning Ways for your Mathematical Plays} desarrolla la teor\'ia de los juegos combinatorios, no solamente hace uso de los n\'umeros surreales sino que trabaja en el concepto m\'as general de juegos. Aunque Conway dice que \textit{On numbers and Games} se escribi\'o en una semana, el libro se demor\'o algo m\'as de 25 a\~nos en publicarse, mientras que recib\'ia correcciones de otros autores y se a\~nad\'ian cap\'itulos.

Por otro lado, el nombre de los n\'umeros surreales se lo di\'o Donald Knuth en su novela de 1974 \textit{Surreal Numbers: how two ex-students turned on to pure mathematics and found total happiness} \cite{Knuth1974-wh}, dos a\~nos antes de que se publicara \textit{On Numbers and Games}. Antes de que Knuth les pusiera el nombre, Conway los llamaba solamente n\'umeros, puesto que todos los n\'umeros estaban descritos en esta clase, tanto los reales como los ordinales.

Nuestro objetivo en este cap\'itulo va a ser definir los n\'umeros surreales y explicar su teor\'ia como cuerpo ordenado. Para esto nos vamos a basar en lo desarrollado en \cite{Conway2000}, \cite{Tondering2019} y \cite{Gonshor1986}.

