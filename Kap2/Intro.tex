Jhon Conway fue un matem\'atico bastante prol\'ifico, seg\'un su p\'agina de Wikipedia \cite{wiki:John_Horton_Conway} Conway trabaj\'o en `teor\'ia de grupos finitos, teor\'ia de nudos, teor\'ia de n\'umeros, teor\'ia combinatoria de juegos, teor\'ia de c\'odigos y matem\'aticas recreacionales', en el libro \textit{The Mathematical Artist} \cite{MathematicalArtist2022} lo describen como un `\textit{polymath}' que transform\'o las matem\'aticas.

Sin embargo, seg\'un su biografa, exist\'ia un descubrimiento que lo enorgullec\'ia m\'as que el resto de sus aportes a las matem\'aticas, los n\'umeros surreales \cite{YTNumberphileConway}. Los n\'umeros surreales son una extensi\'on de los n\'umeros reales que admite infinitos e infinitesimales.

Nuestro objetivo en este cap\'itulo va a ser definir los n\'umeros surreales y explicar su teor\'ia como cuerpo ordenado. Para esto nos vamos a basar en lo desarrollado en \cite{Conway2000}, \cite{Tondering2019} y \cite{Gonshor1986}.

