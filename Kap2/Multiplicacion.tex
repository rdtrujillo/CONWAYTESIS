\section {Multiplicación de números surreales}
    
    La definici\'on de la multiplicaci\'on es mucho m\'as compleja que la definici\'on de la suma. Queremos, igual que en las anteriores definiciones, hacer una definici\'on recursiva a partir de las multiplicaciones de ancestros, y adem\'as, queremos que se respete el orden de los conjuntos $L$ y $R$.

    Una forma de motivarlo es pensar en las propiedades de la multiplicaci\'on real con respecto al orden, la m\'as importante es que la multiplicaci\'on de n\'umeros positivos es positiva. Si pensamos en dos n\'umeros surreales $x,y$, tenemos los n\'umeros positivos
    \begin{align*}
        (x-x^L) > 0, \quad & (x^R-x)>0, \\
        (y-y^L) > 0, \quad & (y^R-y)>0.
    \end{align*}

    Si multiplicamos cada uno de los que corresponden a $x$ con cada uno de los que corresponden a $y$ tendremos $4$ n\'umeros positivos, tomemos como ejemplo dos de ellos, los
    \begin{align*}
        (x-x^L)(y-y^L) &= xy + x^Ly^L - xy^L - x^Ly > 0,\\
        (x-x^L)(y^R-y) &= -xy -x^Ly^R + xy^R + x^Ly > 0,
    \end{align*}
    con los que podemos generar las desigualdades
    \[
        - x^Ly^L + xy^L + x^Ly < xy < -x^Ly^R + xy^R + x^Ly,
    \]
    si hacemos lo mismo con las otras dos posibles multiplicaciones tenemos la definici\'on

    \begin{definition}[Multiplicaci\'on]
        Sean $x, y$ dos n\'umeros surreales. Definimos la multiplicaci\'on de n\'umeros surreales como
        \[
            xy = \big\{x^Ly+xy^L-x^Ly^L, x^Ry+xy^R-x^Ry^R\big|x^Ly+xy^R-x^Ly^R, x^Ry+xy^L-x^Ry^L\big\}.
        \]
    \end{definition}


    \begin{example}
        Para mostrar como funciona la definici\'on primero vamos a ver como funciona para los elementos m\'as sencillos, es decir, para $0$, para $1$ y para $-1$.

        F\'ijese todos los ancestros de $xy$ est\'an hechos a partir de ancestros tanto de $x$ como de $y$, como $0$ no tiene ancestros entonces se cumple que $0x \equiv x0 \equiv \surr{}{} \equiv 0$, igual que en los n\'umeros naturales.
        
        Ahora, mostraremos que $1x \equiv x$. Procederemos por inducci\'on, tenemos que $1\cdot 0 \equiv 0$, por lo tanto para $0$ se cumple. Ahora, supongamos que se cumple para los ancestros de $x$. Tenemos que 
        \[
            1x \equiv \surr{1^Lx + 1x^L - 1^Lx^L}{1^Lx + 1x^R - 1^Lx^R} \equiv \surr{1x^L}{1x^R} \equiv \surr{x^L}{x^R} \equiv x.
        \]

        Por ultimo, mostremos que $-1x \equiv -x$. Demostraremos este hecho por inducci\'on, para $0$ tenemos que $-1\cdot 0\equiv 0 \equiv -0$ por lo tanto se cumple. Ahora, supongamos que se cumple para los ancestros de $x$. Tenemos que 
        \begin{align*}
            -1x & \equiv \surr{(-1)^Rx + (-1)x^R - (-1)^Rx^R}{(-1)^Rx + (-1)x^L - (-1)^Rx^L} \\
             & \equiv \surr{(-1)x^R}{(-1)x^L} \equiv \surr{-x^R}{-x^L} \equiv -x.
        \end{align*}
    \end{example}

    \begin{theorem}[Propiedades de la multiplicaci\'on]
        Sean $x,y,z$ n\'umeros surreales. Se cumplen las siguientes propiedaddes que corresponden a la multiplicaci\'on
        \begin{enumerate}[nosep]
            \item $xy\equiv yx$,
            \item $x(y+z) = xy + xz$,
            \item $x(yz) = (xy)z$.
        \end{enumerate}
    \end{theorem}

    \begin{proof}
        En el ejemplo anterior se pueden ver que todas estas propiedades se cumplen en los casos bases, luego, en esta demostraci\'on solo nos vamos a concentrar en el paso inductivo.

        \begin{enumerate}
            \item Para la conmutatividad tenemos que 
            \begin{align*}
                xy & \equiv \surr{x^Ly+xy^L-x^Ly^L, x^Ry+xy^R-x^Ry^R}{\dots} &\text{(definici\'on)} \\
                & \equiv \surr{yx^L+y^Lx-y^Lx^L, yx^R+y^Rx-y^Rx^R}{\dots} &\text{(h. de inducci\'on)} \\
                & \equiv \surr{y^Lx+yx^L-y^Lx^L, y^Rx+yx^R-y^Rx^R}{\dots} &\text{(conmutatividad +)} \\
                & \equiv yx.
            \end{align*}
            \item Para la propiedad distributiva sobre la suma concentremonos en los t\'erminos de $((x+y)z)^L$ que son de la forma $(x+y)^Lz + (x+y)z^L - (x+y)^Lz^L$, los dem\'as t\'erminos tendr\'an sus desarrollos an\'alogos. En este t\'ermino, el n\'umero $(x+y)^L$ puede ser de dos formas, puede ser o $x+y^L$ o $x^L+y$. Teniendo esto en cuenta tenemos que 
            \begin{align*}
                ((x+y)z) & \equiv \surr{(x+y)^Lz + (x+y)z^L - (x+y)^Lz^L, \dots}{\dots}\\
                & \equiv \big\{(x+y^L)z + (x+y)z^L - (x+y^L)z^L, \\
                & \quad\quad (x^L+y)z + (x+y)z^L - (x^L+y)z^L, \dots\;\big|\;\dots\big\} \\
                & = \surr{xz+(y^Lz + yz^L - y^Lz^L), (x^Lz + xz^L - x^Lz^L) + yz}{}\\
                & \equiv xz + yz.
            \end{align*}
            Aqu\'i no podemos reemplazar la igualdad $(=)$ por la equivalencia $(\equiv)$ puesto que estamos utilizando la propiedad $x + (-x) = 0$ para todo n\'umero surreal $x$. 
            
            Ya que se prob\'o la propiedad distributiva, tenemos que la definici\'on de multiplicaci\'on se puede reescribir como 
            \begin{multline*}
                xy = \big\{xy-(x-x^L)(y-y^L),xy-(x^R-x)(y^R-y)\;\big|\\
                \break xy+(x-x^L)(y^R-y), xy+(x^R-x)(y-y^L)\big\},
            \end{multline*}
            que es m\'as expresiva, adem\'as, tambi\'en combinando con lo que probamos para el n\'umero $-1$, podemos ``distribuir'' los signos en las sumas.
            \item Utilizando la definici\'on discutida en el punto anterior, podemos ver que la multiplicaci\'on de tres elementos es de la forma
            \[
                (xy)z = \surr{(xy)z - \left[(x-x^L)(y-y^L)\right] (z-z^L), \dots}{\dots},
            \]
            f\'ijese que el elemento de $L$ que tenemos escrito se puede escribir solamente en t\'erminos de ancestros de $x$, $y$, y $z$ de la forma
            \[
                (x^Ly)z + (xy^L)z + (xy)z^L - (x^Ly^L)z - (x^Ly)z^L - (xy^L)z^L + (x^Ly^L)z^L,
            \]
            y usando la hip\'otesis te inducci\'on tenemos que el t\'ermino es igual a
            \[
                x^L(yz) + x(y^Lz) + x(yz^L) - x^L(y^Lz) - x^L(yz^L) - x(y^Lz^L) + x^L(y^Lz^L)
            \]
            que reagrupandolo de la misma manera que lo desagrupamos arriba nos queda como 
            \[
                (xy)z = \surr{x(yz) - (x-x^L)\left[(y-y^L)(z-z^L)\right], \dots}{\dots} = x(yz).
            \]
        \end{enumerate}
    \end{proof}
