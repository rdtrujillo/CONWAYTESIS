\chapter{Cap\'{\i}tulo 1}

    Los números surreales se definen de manera recursiva y parecida a las cortaduras de Dedekind. Cada n\'umero surreal es una pareja de conjuntos de n\'umeros surreales a los que se les llama $L$ y $R$ de izquierdo y derecho en ingl\'es, y se representa $\surr{L}{R}$. La idea es que el n\'umero sea mayor que todos los n\'umeros de $L$ y que sea menor que todos los n\'umeros de $R$, que en cierto sentido sea el n\'umero m\'as `sencillo' que est\'a en la mitad de los dos conjuntos.

    El primer n\'umero que se crea de esta manera es el n\'umero comformado por la pareja $L = \emptyset$ y $R=\emptyset$, es decir, $\surr{\emptyset}{\emptyset}$. A este n\'umero se le llama $0$, y veremos luego que tiene las mismas propiedades del $0$ de los n\'umeros reales.

    Ya teniendo el $0$, se pueden formar las parejas de conjuntos
    \[
        \surr{\{0\}}{\emptyset}, \quad \surr{\emptyset}{\{0\}}, \quad \surr{\{0\}}{\{0\}}.
    \]
    
    Para hacer la notaci\'on m\'as sencilla, vamos a escribir solamente los elementos de $L$ y $R$ sin los corchetes, as\'i, tendremos que $0=\surr{\emptyset}{\emptyset} = \surr{}{}$, y otro ejemplo, $\surr{\{0\}}{\emptyset} = \surr{0}{}$, porque tambi\'en luego veremos que tienen las mismas propiedades que sus contrapartes reales.

    Si queremos que el n\'umero est\'e entre $L$ y $R$ entonces tendremos que todos los elementos de $L$ tienen que ser `menores' que todos los elementos de $R$, aunque a\'un no hayamos definido un orden en el conjunto. De este modo podemos ver que el par $\surr{0}{0}$ no puede ser un n\'umero surreal ya que $L$ y $R$ comparten el mismo elemento.

    Las otras dos parejas que quedan, $\surr{0}{}$ y $\surr{}{0}$, s\'i son n\'umeros surreales y tienen su propio nombre. Decimos que $1 :=\surr{0}{}$ y $-1 :=\surr{}{0}$. Luego podremos ver que efectivamente estos dos n\'umeros tienen las mismas propiedades que sus correspondientes n\'umeros reales.

    Si seguimos haciendo nuevas parejas con los n\'umeros que ya creamos tendremos entonces las parejas
    \begin{align*}
        &\surr{1}{}, \surr{}{1},\\
        &\surr{-1}{}, \surr{}{-1}, \\
        &\surr{0,1}{}, \surr{0}{1}, \surr{}{0,1} \\
        &\surr{-1,0}{}, \surr{-1}{0}, \surr{}{-1,0} \\
        &\surr{-1,1}{}, \surr{-1}{1}, \surr{}{-1,1} \\
        &\surr{-1,0,1}{}, \surr{-1,0}{1}, \surr{-1}{0,1}, \surr{}{-1,0,1},
    \end{align*}
    y podremos seguir haciendo m\'as n\'umeros con estos nuevos n\'umeros.
    
    Como convención, si tenemos un número surreal $x=\{L|R\}$, entonces llamaremos $x^L$ a los elementos de $L$ y al conjunto $L$ lo llamaremos $X^L$ en may\'uscula; tambi\'en llamaremos $x^R$ a los elementos de $R$ y al conjunto $R$ lo llamaremos $X^R$ en may\'uscula.

    Hasta ahora hemos hablado de una relaci\'on de orden sin definirla, lo hemos hecho para que se pueda ver primero el car\'acter recursivo de los n\'umeros surreales. El orden tambi\'en tiene una definici\'on igualmente recursiva, lo que hacemos es intentar definir el orden con los elementos de los conjuntos $L$ y $R$, es decir, para los $x^L$ y $x^R$.

    Para motivar las definiciones formales, tanto de n\'umeros surreales como de orden, intentemos pensar en un n\'umero surreal $x = \surr{L}{R}$. Este orden va a terminar siendo un orden lineal, igual que en los n\'umeros reales, entonces cuando decimos que
    \[
        x^L < x^R
    \]
    para todos los elementos de los conjuntos $X^L$ y $X^R$, estamos diciendo equivalentemente que
    \[
        x^R \not\le x^L.
    \]
    De esta forma, la propiedad de que los elementos de $X^L$ tienen que ser todos menores que los elementos de $X^R$ queda definido en base de la relaci\'on de orden.

    \begin{definition}[N\'umero surreal]
        Sea $x$ una pareja de conjuntos de n\'umeros surreales $x=\surr{L}{R}$. Se dice que $x$ es un \textbf{n\'umero surreal}, si y solamente si, se tiene que ningún $x^R$ es menor o igual ($\le$) que algún $x^L$. 
    \end{definition}

    Si tenemos dos n\'umeros surreales $x$ y $y$ tales que $x\le y$ y queremos definir esta relaci\'on en base a los elementos de sus conjuntos $L$ y $R$ podemos tener en cuenta las desigualdades
    \[
        x^L < x < x^R,\quad y^L < y < y^R.
    \]
    Estas desigualdades juntas con el hecho de que $x\le y$ generan las desigualdades
    \[
        x < y^R,\quad x^L < y,
    \]
    que est\'an basadas en elementos m\'as `simples', es decir, los elementos de los conjuntos $L$ y $R$ de $x$ y $y$.

    \begin{definition}[Orden en números surreales]
        Sean $x$ y $y$ dos n\'umeros surreales. Se dice que $x\le y$, si y solamente si, se tiene que ning\'un $x^L$ es mayor o igual $(\ge)$ que $y$ y ningún $y^R$ es menor o igual $(\le)$ a $x$.
    \end{definition}

    Igual que en la definici\'on de n\'umero surreal se hace uso de la negaci\'on de el orden estricto para que la definici\'on quede en t\'erminos de la relaci\'on de orden. Adem\'as se sigue la misma convención que se utiliza para las relaciones de orden, es decir, $x\le y$ es equivalente a decir que $y\ge x$, adem\'as si tenemos que $x\le y$ y $y\le x$ se dice que $x=y$. Tambi\'en se dice que si $x\le y$ pero $x \not= y$, entonces se escribe $x < y$, igualmente para $x > y$.

    En este momento podemos probar que efectivamente los n\'umeros que conocemos son en efecto n\'umeros surreales.

    \begin{theorem}
        $0=\surr{}{}$ es un n\'umero surreal.
    \end{theorem}

    \begin{proof}
        Tenemos que probar que ning\'un elemento $0^L$ es mayor o igual que algún elemento de $0^R$, pero como tanto $L$ como $R$ son vac\'ios en $0$ entonces esto se cumple por vacuidad. 
    \end{proof}

    \begin{theorem}
        $1=\surr{0}{}$ y $-1=\surr{}{1}$ son n\'umeros surreales.
    \end{theorem}

    \begin{proof}
        Probemos primero para $1=\surr{0}{}$. Tenemos que probar que ning\'un elemento $1^L$ es mayor o igual que algún elemento de $1^R$, pero f\'ijese que no hay ning\'un elemento en $R$, por lo tanto tendremos que tambi\'en se cumple por vacuidad.

        Un caso parecido lo tendremos para $-1$, pero en este caso con el conjunto $L$.
    \end{proof}

    Un ejemplo m\'as de como se puede usar esta definici\'on de orden es probar las desigualdades esperadas $-1< 0$ y $0<1$.

    \begin{theorem}
        $-1 < 0$ y $0 < 1$.
    \end{theorem}

    \begin{proof}
        Probemos $0 < 1$, la otra desigualdad se hace de manera parecida. Primero probemos que efectivamente $0\le 1$. Tenemos que probar que no hay elementos en $L$ y $R$ de los respectivos n\'umeros tales que
        \[
            0 \ge 1^R, \quad 0^L \ge 1,
        \]
        pero f\'ijese que no hay ningún elemento $1^R$, ni tampoco ningún elemento $0^L$, por lo tanto la propiedad se cumple por vacuidad.

        Ahora probemos que $0 \not\ge 1$. F\'ijese que en $1^L$ est\'a el elemento $0$, y efectivamente $0 = 1^L \ge 0$, por lo que tenemos que $0 < 1$.
    \end{proof}

    Algo que a\'un no hemos probado pero hemos inferido es que $\le$ es una relaci\'on de orden, es decir, que es reflexiva y transitiva. La prueba de esto utiliza fuertmente la naturaleza recursiva de la definici\'on de orden y es bastante parecida a muchas pruebas que se van a utilizar en el futuro cuando definamos las operaciones de los n\'umeros. B\'asicamente, nuestra hip\'otesis de inducci\'on funcionar\'a sobre los elementos de $L$ y $R$ de nuestro n\'umero, y nuestro caso base siempre ser\'a cuando el n\'umero sea $0$ ya que todos los n\'umeros son descendientes de este.

    \begin{theorem}[Reflexividad]
        Para todo n\'umero surreal $x$, se tiene que $x\le x$.
    \end{theorem}

    \begin{proof}
        F\'ijese que para $0$ se tiene la propiedad porque los conjuntos $L$ y $R$ de $0$ son vac\'ios.

        Sea $x$ un n\'umero surreal y supongamos que la propiedad se cumple para todo $x^L$ y todo $x^R$. Tenemos que probar que $x\le x$, es decir, para ning\'un elemento de $X^L$ y $X^R$ se tiene que
        \[
            x \le x^L, \quad x^R \le x.
        \]
        Veamos la primera parte, es decir, que $x \not\le x^L$. Esto quiere decir por definici\'on que existe un elemento\footnote{Ac\'a usamos la notaci\'on $(X^L)^R$ para referirnos al conjunto $R$ del elemento $x^L$ que hab\'iamos descrito.} $y\in (X^L)^R$  tal que $x \ge y$, o que existe un elemento $y \in X^L$ tal que $y \le x^L$. F\'ijese que si tomamos $y = x^L \in X^L$ entonces tenemos que $y = x^L \le x^L$, por la hip\'otesis te inducci\'on.

        Para la segunda parte, es decir, que $x^R \not\le x$, se hace un an\'alisis parecido.
    \end{proof}
    
    \begin{corollary}
        Para todo n\'umero surreal $x$, $x = x$.
    \end{corollary}

    \begin{proof}
        Utilizamos la reflexividad de la relaci\'on de orden.
    \end{proof}

    Para demostrar la transitividad necesitaremos utilizar la inducci\'on pero en triplas de n\'umeros surreales, es decir, vamos a demostrar la transitividad para $(x,y,z)$ pero usando la hip\'otesis para las triplas que contengan alguno de los elementos de los conjuntos $L$ y $R$ de $x,y$ o $z$. Como siempre estaremos preguntando sobre la propiedad en alguno de los elementos de los $L$ y $R$, al final llegaremos a preguntarlo en el conjunto $(0,0,0)$ que ser\'a nuestro caso base.

    \begin{theorem}[Transitividad]
        Sean $x,y,z$ n\'umeros surreales. Si tenemos que $x\le y$ y $y\le z$, entonces se tiene que $x\le z$.
    \end{theorem}

    \begin{proof}
        Utilicemos inducci\'on sobre las triplas $(x,y,z)$. F\'ijese que para la tripla $(0,0,0)$ la propiedad se cumple gracias a la reflexividad.

        Ahora, supongamos por inducci\'on que se cumple para todas las triplas con alg\'un elemento de los conjuntos $L$ y $R$ de $(x,y,z)$ y adem\'as supongamos que $x\le y$ y $y\le z$. Supongamos por contradicci\'on que $x\not\le z$, esto es, existe un elemento $z^R \le x$ o existe un elemento $x^L \ge z$.

        Veamos el primer caso, tenemos que $z^R \le x$. Como $x \le y$ por hip\'otesis de inducci\'on tenemos qeu $z^R \le y$, por lo tanto, por la definici\'on de orden tendremos que $y \not\le z$, contradicci\'on.

        Ahora veamos el segundo caso, tenemos que $x^L \ge z$. Como $z \ge y$ por hip\'otesis de inducci\'on tenemos que $x^L \ge y$, por lo tanto, por la definici\'on de orden tendremos que $x\not\le y$.

        En cualquier caso es una contradicci\'on, entonces la relaci\'on $\le$ es transitiva.
    \end{proof}

    \begin{corollary}
        Sean $x,y,z$ n\'umeros surreales. Si $x = y$ y $y = z$, entonces $x = z$.
    \end{corollary}

    Con esto ya podemos decir que la relaci\'on es de orden, y adem\'as tambi\'en tenemos una relaci\'on de equivalencia entre n\'umeros surreales.

    Una pregunta que nos podr\'iamos hacer en este momento es sobre si existen distintas representaciones de un mismo n\'umero, es decir: ¿Existen dos n\'umeros $x$ y $y$ tales que $x=y$ pero que sus conjuntos $L$ y $R$ sean diferentes?

    La respuesta la tenemos en varias de los ejemplos que ya tenemos sobre n\'umeros surreales, es m\'as, tenemos que
    \begin{align*}
        &\surr{}{-1} = \surr{}{-1,0} = \surr{}{-1,1} = \surr{}{-1,0,1}, \\
        &-1 = \surr{}{0,1},\\
        &\surr{-1}{1} = \surr{-1}{1,0},\\
        &0 = \surr{-1}{1} = \surr{-1}{} = \surr{}{1},\\
        &\surr{-1,0}{1} = \surr{0}{1},\\
        &1 = \surr{-1,0}{},\\
        &\surr{1}{} = \surr{0,1}{} = \surr{-1,1}{} = \surr{-1,0,1}{}.
    \end{align*}

    Muchos de estos ejemplos se pueden ver teniendo en cuenta que agregar n\'umeros menores que los que est\'an en el conjunto $L$ al conjunto $L$ genera el mismo n\'umero, e igualmente agregar n\'umeros mayores al conjunto $R$.

    Una propiedad que hemos estado referenciando sin demostrar para motivar las definiciones es la idea de que el n\'umero que definimos est\'a entre los elementos de $L$ y los elementos de $R$, m\'as espec\'ificamente

    \begin{theorem}
        Sea $x$ un n\'umero surreal. Tenemos que $x^L < x < x^R$.
    \end{theorem}

    \begin{proof}
        Probemos primero que $x^L \le x \le x^R$ y luego hacemos la desigualdad estricta. Hagamos la prueba para la parte izquierda de la desigualdad, la parte derecha se hace de manera parecida.

        Probemos que $x^L \le x$ por inducci\'on. La propiedad es verdadera para $0$ por vacuidad. Ahora, supongamos que es verdad para $x^L$ y probemos para $x$. Tenemos que probar que no existe ning\'un elemento $x^R \le x^L$ para todos los elementos $x^R$ y tampoco existe ningún elemento $y\in (X^L)^L$ tal que $y \ge x$. La primera parte la tenemos porque $x$ es un n\'umero surreal entonces ning\'un elemento de $X^L$ es mayor que ning\'un elemento de $X^R$. Para la segunda parte, probemos que $y \not\ge x$, esto significa que o existe $z\in X^L$ tal que $y\le z$, o existe $y^R$ tal que $y^R\le x$. F\'ijese que si tomamos $z = x^L$ entonces tendremos por hip\'otesis de inducci\'on que $y\le z = x^L$, puesto que $y\in (X^L)^L$, luego tenemos que $x^L\le x$.

        Ahora, para hacer la desigualdad estricta f\'ijese que $x^L \not\ge x$ significa que existe $y\in X^L$ tal que $y\ge x^L$, o que existe $z\in (X^L)^R$ tal que $z\le x$, si tomamos $y=x^L$ tendremos por reflexividad que $x^L\not\ge x$.
    \end{proof}

    \begin{corollary}[Linealidad]
        Sean $x, y$ n\'umeros surreales. Si $x \not\le y$ entonces tenemos que $x > y$. 
    \end{corollary}

    \begin{proof}
        Supongamos que $x\not\le y$. Esto quiere decir que o existe $x^L \ge y$, o existe $y^R \le x$. Si existe $x^L \ge y$, entonces como $x^L < x$ por transitividad $x > y$. Si existe $y^R \le x$, entonces como $y < y^R$ por transitividad $y < x$.
    \end{proof}

    La suma de los números surreales también se define recursivamente, se dice que
    \[
        x + y  = \left\{x^L+y, x+y^L\;|\;x^R+y, x+y^R\right\}.
    \]

    Esta suma deberia cumplir las propiedades usuales de la suma en numeros reales, por ejemplo, que $z+0=z$ para todo numero surreal. Y efectivamente, probemos la proposicion por induccion. Tenemos que $0+0 = 0$ porque los conjuntos $L$ y $R$ de $0$ son vacios. Ahora, por la definicion de suma $z+0 = \left\{z^L+0|z^R+0\right\}$ y por hipotesis de induccion tendremos que $z+0 = \left\{z^L|z^R\right\} = z$.

    Para todo numero surreal existe tambien su inverso aditivo tal que al ser sumado con el el resultado es $0$. Para construir el inverso aditivo podemos tambien hacer una construccion recursiva, es decir, si $x=\{x^L|x^R\}$ entonces tomamos 
    \[
        -x = \{-x^R|-x^L\}.
    \]

    Un ejemplo son los numeros $\{0|\}$ y $\{|0\}$, que al sumarse da el numero $x = \{0|\} + \{|0\} = \{\{|0\}|\{0|\}\}$. Fijese que $x^L = \{|0\}\le 0$ y $x^R = \{0|\}\ge 0$. Por lo tanto podemos comprobar que $x\le 0$ y que $x \ge 0$, es decir, efectivamente son inversos aditivos.

    La multiplicacion de numeros tambien se define recursivamente, se dice que
    \[
        xy = \{x^Ly+xy^L-x^Ly^L, x^Ly+xy^R-x^Ry^R| x^Ly+xy^R-x^Ly^R, x^Ry+xy^L-x^Ry^L\}.
    \]

    Esta definicion un tanto mas compleja que la de la suma sale de la multiplicacion de las desigualdades
    \begin{align*}
        (x-x^L) > 0, \quad & (x^R-x)>0 \\
        (y-y^L) > 0, \quad & (y^R-y)>0.
    \end{align*}
    