\chapter{Números Surreales}

    Los números surreales se definen de manera recursiva y parecida a las cortaduras de Dedekind\footnote{En las cortaduras de Dedekind, se definen los n\'umeros reales como parejas de conjuntos $L$ y $R$ de n\'umeros racionales con la idea de que el n\'umero que est\'a representando sea mayor o igual que todos los elementos de $L$ y menor o igual que todos los elementos de $R$.}. Cada n\'umero surreal es una pareja de conjuntos de n\'umeros surreales a los que se les llama $L$ y $R$ de izquierdo y derecho en ingl\'es, y se representa $\surr{L}{R}$. La idea es que el n\'umero sea ``mayor'' que todos los n\'umeros de $L$ y que sea ``menor'' que todos los n\'umeros de $R$, que en cierto sentido sea el n\'umero m\'as `sencillo' que est\'a en la mitad de los dos conjuntos.

    Como la definici\'on es recursiva, empezamos con el conjunto m\'as sencillo posible: el conjunto vac\'io. El primer n\'umero que se crea de esta manera es el n\'umero comformado por la pareja $L = \emptyset$ y $R=\emptyset$, es decir, $\surr{\emptyset}{\emptyset}$. A este n\'umero se le llama $0$, y veremos luego que tiene las mismas propiedades del $0$ de los n\'umeros reales, esto es, que es el m\'odulo de la suma en n\'umeros surreales y que cualquier n\'umero multiplicado por este da $0$.

    Ya teniendo el $0$, se pueden formar las parejas de conjuntos
    \[
        \surr{\{0\}}{\emptyset}, \quad \surr{\emptyset}{\{0\}}, \quad \surr{\{0\}}{\{0\}}.
    \]
    
    Para hacer la notaci\'on m\'as sencilla, vamos a escribir solamente los elementos de $L$ y $R$ sin los corchetes, as\'i, tendremos que $0=\surr{\emptyset}{\emptyset} = \surr{}{}$, y por ejemplo, $\surr{\{0\}}{\emptyset} = \surr{0}{}$, porque tambi\'en veremos que tienen las mismas propiedades que sus contrapartes reales.

    Si queremos que el n\'umero est\'e entre $L$ y $R$ entonces tendremos que todos los elementos de $L$ tienen que ser `menores' que todos los elementos de $R$, aunque a\'un no hayamos definido un orden en el conjunto. De este modo podemos ver que el par $\surr{0}{0}$ no puede ser un n\'umero surreal ya que $L$ y $R$ comparten el mismo elemento.

    Las otras dos parejas que quedan, $\surr{0}{}$ y $\surr{}{0}$, s\'i son n\'umeros surreales y tienen su propio nombre. Decimos que $1 :=\surr{0}{}$ y $-1 :=\surr{}{0}$. Luego podremos ver que efectivamente estos dos n\'umeros tienen las mismas propiedades que sus correspondientes n\'umeros reales.

    Si seguimos haciendo nuevas parejas con los n\'umeros que ya creamos tendremos entonces las parejas
    \begin{align*}
        &\surr{1}{}, \surr{}{1},\\
        &\surr{-1}{}, \surr{}{-1}, \\
        &\surr{0,1}{}, \surr{0}{1}, \surr{}{0,1} \\
        &\surr{-1,0}{}, \surr{-1}{0}, \surr{}{-1,0} \\
        &\surr{-1,1}{}, \surr{-1}{1}, \surr{}{-1,1} \\
        &\surr{-1,0,1}{}, \surr{-1,0}{1}, \surr{-1}{0,1}, \surr{}{-1,0,1},
    \end{align*}
    y podremos seguir haciendo m\'as n\'umeros con estos nuevos n\'umeros.
    
    Como convención, si tenemos un número surreal $x=\{L|R\}$, entonces llamaremos $x^L$ a los elementos de $L$ y al conjunto $L$ lo llamaremos $X^L$ en may\'uscula; tambi\'en llamaremos $x^R$ a los elementos de $R$ y al conjunto $R$ lo llamaremos $X^R$ en may\'uscula.

    Hasta ahora hemos hablado de una relaci\'on de orden sin definirla, lo hemos hecho para que se pueda ver primero el car\'acter recursivo de los n\'umeros surreales. El orden tambi\'en tiene una definici\'on igualmente recursiva, lo que hacemos es intentar definir el orden con los elementos de los conjuntos $L$ y $R$, es decir, para los $x^L$ y $x^R$.

    Para motivar las definiciones formales, tanto de n\'umeros surreales como de orden, intentemos pensar en un n\'umero surreal $x = \surr{L}{R}$. Este orden va a terminar siendo un orden total, igual que en los n\'umeros reales, entonces cuando decimos que
    \[
        x^L < x^R
    \]
    para todos los elementos de los conjuntos $X^L$ y $X^R$ respectivamente, estamos diciendo equivalentemente que
    \[
        x^R \not\le x^L.
    \]
    De esta forma, usando el orden $\le$ tendremos que todos los elementos de $X^L$ deben ser menores que los elementos de $X^R$.

    \begin{definition}[N\'umero surreal]
        Sea $x$ una pareja de conjuntos de n\'umeros surreales $x=\surr{L}{R}$. Se dice que $x$ es un \textbf{n\'umero surreal}, si y solamente si, se tiene que ningún $x^R$ es menor o igual ($\le$) que algún $x^L$. 
    \end{definition}

    Si tenemos dos n\'umeros surreales $x$ y $y$ tales que $x\le y$ y queremos definir esta relaci\'on en base a los elementos de sus conjuntos $L$ y $R$ podemos tener en cuenta las desigualdades
    \[
        x^L < x < x^R,\quad y^L < y < y^R.
    \]
    Estas desigualdades juntas con el hecho de que $x\le y$ generan las desigualdades
    \[
        x < y^R,\quad x^L < y,
    \]
    que est\'an basadas en elementos m\'as `simples', es decir, los elementos de los conjuntos $L$ y $R$ de $x$ y $y$.

    \begin{definition}[Orden en números surreales]
        Sean $x$ y $y$ dos n\'umeros surreales. Se dice que $x\le y$, si y solamente si, se tiene que ning\'un $x^L$ es mayor o igual $(\ge)$ que $y$ y ningún $y^R$ es menor o igual $(\le)$ a $x$.
    \end{definition}

    Igual que en la definici\'on de n\'umero surreal se hace uso de la negaci\'on de la relaci\'on de orden. Adem\'as se sigue la misma convención que se utiliza para las relaciones de orden, es decir, $x\le y$ es equivalente a decir que $y\ge x$, adem\'as si tenemos que $x\le y$ y $y\le x$ se dice que $x=y$. Tambi\'en se dice que si $x\le y$ pero $x \not= y$, entonces se escribe $x < y$, igualmente para $x > y$.

    En este momento podemos probar que efectivamente los n\'umeros que conocemos son en efecto n\'umeros surreales.

    \begin{theorem}
        $0=\surr{}{}$ es un n\'umero surreal.
    \end{theorem}

    \begin{proof}
        Tenemos que probar que ning\'un elemento $0^L$ es mayor o igual que algún elemento de $0^R$, pero ya que tanto $L$ como $R$ son vac\'ios entonces esto se cumple por vacuidad. 
    \end{proof}

    \begin{theorem}
        $1=\surr{0}{}$ y $-1=\surr{}{0}$ son n\'umeros surreales.
    \end{theorem}

    \begin{proof}
        Probemos primero para $1=\surr{0}{}$. Tenemos que probar que ning\'un elemento $1^L$ es mayor o igual que algún elemento de $1^R$, pero f\'ijese que no hay ning\'un elemento en $R$, por lo tanto tendremos que tambi\'en se cumple por vacuidad.

        Para $-1$ es algo parecido. Tenemos que probas que ning\'un elemento $(-1)^L$ es mayor o igual que alg\'un elemento de $(-1)^R$, pero tenemos que $L$ es vaci\'o, por lo tanto tendremos que tambi\'en se cumple por vacuidad.
    \end{proof}

    Un ejemplo m\'as de como se puede usar esta definici\'on de orden es probar las desigualdades esperadas $-1< 0$ y $0<1$.

    \begin{theorem}
        $-1 < 0$ y $0 < 1$.
    \end{theorem}

    \begin{proof}
        Probemos $0 < 1$. Primero probemos que efectivamente $0\le 1$. Tenemos que probar que no hay elementos en $L$ y $R$ de los respectivos n\'umeros tales que
        \[
            0 \ge 1^R, \quad 0^L \ge 1,
        \]
        pero f\'ijese que no hay ningún elemento $1^R$, ni tampoco ningún elemento $0^L$, por lo tanto la propiedad se cumple por vacuidad.

        Ahora probemos que $0 \not\ge 1$. Esto es, queremos probar que existe alg\'un elemento F\'ijese que en $1^L$ est\'a el elemento $0$, y efectivamente $0 = 1^L \ge 0$, por lo que tenemos que $0 < 1$.
    \end{proof}

    Algo que a\'un no hemos probado pero hemos inferido es que $\le$ es una relaci\'on de orden, es decir, que es reflexiva y transitiva. La prueba de esto utiliza fuertmente la naturaleza recursiva de la definici\'on de orden y es bastante parecida a muchas pruebas que se van a utilizar en el futuro cuando definamos las operaciones de los n\'umeros. B\'asicamente, nuestra hip\'otesis de inducci\'on funcionar\'a sobre los elementos de $L$ y $R$ de nuestro n\'umero, y nuestro caso base siempre ser\'a cuando el n\'umero sea $0$ ya que todos los n\'umeros son descendientes de este.

    \begin{theorem}[Reflexividad]
        Para todo n\'umero surreal $x$, se tiene que $x\le x$.
    \end{theorem}

    \begin{proof}
        F\'ijese que para $0$ se tiene la propiedad porque los conjuntos $L$ y $R$ de $0$ son vac\'ios.

        Sea $x$ un n\'umero surreal y supongamos que la propiedad se cumple para todo $x^L$ y todo $x^R$. Tenemos que probar que $x\le x$, es decir, para ning\'un elemento de $X^L$ y $X^R$ se tiene que
        \[
            x \le x^L, \quad x^R \le x.
        \]
        Veamos la primera parte, es decir, que $x \not\le x^L$. Esto quiere decir por definici\'on que existe un elemento\footnote{Ac\'a usamos la notaci\'on $(X^L)^R$ para referirnos al conjunto $R$ del elemento $x^L$ que hab\'iamos descrito.} $y\in (X^L)^R$  tal que $x \ge y$, o que existe un elemento $y \in X^L$ tal que $y \le x^L$. F\'ijese que si tomamos $y = x^L \in X^L$ entonces tenemos que $y = x^L \le x^L$, por la hip\'otesis te inducci\'on.

        Para la segunda parte, es decir, que $x^R \not\le x$, se hace un an\'alisis parecido.
    \end{proof}
    
    \begin{corollary}
        Para todo n\'umero surreal $x$, $x = x$.
    \end{corollary}

    \begin{proof}
        Utilizamos la reflexividad de la relaci\'on de orden.
    \end{proof}

    Para demostrar la transitividad necesitaremos utilizar la inducci\'on pero en triplas de n\'umeros surreales, es decir, vamos a demostrar la transitividad para $(x,y,z)$ pero usando la hip\'otesis para las triplas que contengan alguno de los elementos de los conjuntos $L$ y $R$ de $x,y$ o $z$. Como siempre estaremos preguntando sobre la propiedad en alguno de los elementos de los $L$ y $R$, al final llegaremos a preguntarlo en el conjunto $(0,0,0)$ que ser\'a nuestro caso base.

    \begin{theorem}[Transitividad]
        Sean $x,y,z$ n\'umeros surreales. Si tenemos que $x\le y$ y $y\le z$, entonces se tiene que $x\le z$.
    \end{theorem}

    \begin{proof}
        Utilicemos inducci\'on sobre las triplas $(x,y,z)$. F\'ijese que para la tripla $(0,0,0)$ la propiedad se cumple gracias a la reflexividad.

        Ahora, supongamos por inducci\'on que se cumple para todas las triplas con alg\'un elemento de los conjuntos $L$ y $R$ de $(x,y,z)$ y adem\'as supongamos que $x\le y$ y $y\le z$. Supongamos por contradicci\'on que $x\not\le z$, esto es, existe un elemento $z^R \le x$ o existe un elemento $x^L \ge z$.

        Veamos el primer caso, tenemos que $z^R \le x$. Como $x \le y$ por hip\'otesis de inducci\'on tenemos qeu $z^R \le y$, por lo tanto, por la definici\'on de orden tendremos que $y \not\le z$, contradicci\'on.

        Ahora veamos el segundo caso, tenemos que $x^L \ge z$. Como $z \ge y$ por hip\'otesis de inducci\'on tenemos que $x^L \ge y$, por lo tanto, por la definici\'on de orden tendremos que $x\not\le y$.

        En cualquier caso es una contradicci\'on, entonces la relaci\'on $\le$ es transitiva.
    \end{proof}

    \begin{corollary}
        Sean $x,y,z$ n\'umeros surreales. Si $x = y$ y $y = z$, entonces $x = z$.
    \end{corollary}

    Con esto ya podemos decir que la relaci\'on es de orden, y adem\'as tambi\'en tenemos una relaci\'on de equivalencia entre n\'umeros surreales.

    Una pregunta que nos podr\'iamos hacer en este momento es sobre si existen distintas representaciones de un mismo n\'umero, es decir: ¿Existen dos n\'umeros $x$ y $y$ tales que $x=y$ pero que sus conjuntos $L$ y $R$ sean diferentes?

    La respuesta la tenemos en varias de los ejemplos que ya tenemos sobre n\'umeros surreales, es m\'as, tenemos que
    \begin{align*}
        &\surr{}{-1} = \surr{}{-1,0} = \surr{}{-1,1} = \surr{}{-1,0,1}, \\
        &-1 = \surr{}{0,1},\\
        &\surr{-1}{1} = \surr{-1}{1,0},\\
        &0 = \surr{-1}{1} = \surr{-1}{} = \surr{}{1},\\
        &\surr{-1,0}{1} = \surr{0}{1},\\
        &1 = \surr{-1,0}{},\\
        &\surr{1}{} = \surr{0,1}{} = \surr{-1,1}{} = \surr{-1,0,1}{}.
    \end{align*}

    Muchos de estos ejemplos se pueden ver teniendo en cuenta que agregar n\'umeros menores que los que est\'an en el conjunto $L$ al conjunto $L$ genera el mismo n\'umero, e igualmente agregar n\'umeros mayores al conjunto $R$.

    Una propiedad que hemos estado referenciando sin demostrar para motivar las definiciones es la idea de que el n\'umero que definimos est\'a entre los elementos de $L$ y los elementos de $R$, m\'as espec\'ificamente

    \begin{theorem}
        Sea $x$ un n\'umero surreal. Tenemos que $x^L < x < x^R$.
    \end{theorem}

    \begin{proof}
        Probemos primero que $x^L \le x \le x^R$ y luego hacemos la desigualdad estricta. Hagamos la prueba para la parte izquierda de la desigualdad, la parte derecha se hace de manera parecida.

        Probemos que $x^L \le x$ por inducci\'on. La propiedad es verdadera para $0$ por vacuidad. Ahora, supongamos que es verdad para $x^L$ y probemos para $x$. Tenemos que probar que no existe ning\'un elemento $x^R \le x^L$ para todos los elementos $x^R$ y tampoco existe ningún elemento $y\in (X^L)^L$ tal que $y \ge x$. La primera parte la tenemos porque $x$ es un n\'umero surreal entonces ning\'un elemento de $X^L$ es mayor que ning\'un elemento de $X^R$. Para la segunda parte, probemos que $y \not\ge x$, esto significa que o existe $z\in X^L$ tal que $y\le z$, o existe $y^R$ tal que $y^R\le x$. F\'ijese que si tomamos $z = x^L$ entonces tendremos por hip\'otesis de inducci\'on que $y\le z = x^L$, puesto que $y\in (X^L)^L$, luego tenemos que $x^L\le x$.

        Ahora, para hacer la desigualdad estricta f\'ijese que $x^L \not\ge x$ significa que existe $y\in X^L$ tal que $y\ge x^L$, o que existe $z\in (X^L)^R$ tal que $z\le x$, si tomamos $y=x^L$ tendremos por reflexividad que $x^L\not\ge x$.
    \end{proof}

    \begin{corollary}[Linealidad]
        Sean $x, y$ n\'umeros surreales. Si $x \not\le y$ entonces tenemos que $x > y$. 
    \end{corollary}

    \begin{proof}
        Supongamos que $x\not\le y$. Esto quiere decir que o existe $x^L \ge y$, o existe $y^R \le x$. Si existe $x^L \ge y$, entonces como $x^L < x$ por transitividad $x > y$. Si existe $y^R \le x$, entonces como $y < y^R$ por transitividad $y < x$.
    \end{proof}

    Un problema se nos va a presentar de ahora en adelante cuando intentemos definir operaciones en n\'umeros surreales; tendremos que ver si estas operaciones son compatibles con la relaci\'on de equivalencia generada por la relaci\'on de orden.

    Incluso algo que nos podemos preguntar es si nuestra relaci\'on de orden es en `compatible' con nuestra definici\'on de n\'umero. Con esto me refiero a si puedo cambiar los elementos del $L$ y el $R$ por elementos iguales sin cambiar la clase de equivalencia del n\'umero, m\'as formalmente

    \begin{theorem}
        Sean $L,L',R, R'$ conjuntos de n\'umeros surreales tales que todos los elementos de $L$ son iguales que alg\'un elemento de $L'$, todos los elementos de $R$ son iguales que alg\'un elemento de $R'$, y viceversa. Adem\'as supongamos que $\surr{L}{R}$ y $\surr{L'}{R'}$ son n\'umeros surreales. Tenemos que $\surr{L}{R} = \surr{L'}{R'}$.
    \end{theorem}

    \begin{proof}
        Mostremos que en efecto $\surr{L}{R} = \surr{L'}{R'}$. Para esto es suficiente mostrar que $\surr{L}{R} \le \surr{L'}{R'}$  puesto que la otra desigualdad se sigue por simetr\'ia.

        Llamemos $x = \surr{L}{R}$ y $x' = \surr{L'}{R'}$. Tenemos que probar que $x^L < x'$ y $x < (x')^R$. Como existe en $L'$ un elemento $y$ tal que $x^L = y$, entonces tenemos que $x^L = y < x'$, por otro lado, en $R$ existe un elemento $z$ tal que $(x')^R = z$, por lo tanto tenemo que $x < z = (x')^R$, con lo que tendr\'iamos el resultado.
    \end{proof}

    \section{Suma de números surreales}

    La suma de los números surreales también se define recursivamente.
    
    \begin{definition}[Definici\'on de suma]
        Se dice que
        \[
            x + y  = \left\{x^L+y, x+y^L\;|\;x^R+y, x+y^R\right\}.
        \]
        Esta definici\'on hace que la suma sea autom\'aticamente conmutativa ya que es lo mismo en definici\'on hacer $x+y$ o $y+x$.
    \end{definition}

    Como todas las definiciones recursivas que hemos hecho, la suma se define eventualmente en base al número $0$, por eso es bueno ver qué pasa cuando se suma $0$.

    \begin{example}
        ¿Qué pasa cuando se suma $0+0$? Sabemos que $0 = \surr{}{}$, por lo tanto, no existe ni $0^L$ ni $0^R$. Lo que tendremos entonces es
        \[
            0+0 = \surr{0^L+0, 0+0^L}{0^R+0, 0+0^R} = \surr{}{} = 0,
        \]
        por lo tanto tendremos que $0+0 = 0$.
    \end{example}

    \begin{example}
        Más aún, si el $0$ de los números surreales se parece al $0$ de los números reales entonces tendríamos que el $0$ es módulo de la suma, esto es, $x+0 = x$ para todo $x$ n\'umero surreal.

        Mostremos esto por inducci\'on. Nuestro caso base es cuando $x = 0$, es decir, $0+0$ que ya probamos que $0+0=0=x$ luego se cumple.

        Supongamos que se cumple para todos los elementos en los conjuntos $X^L$ y $X^R$. Tenemos entonces que
        \[
            x + 0 = \surr{x^L+0, x+0^L}{x^R+0, x+0^L} = \surr{x^L+0}{x^R+0}
        \]
        y por hip\'otesis de inducci\'on tenemos que $x^L+0 = x^L$ y $x^R+0 = x^R$, por lo tanto 
        \[
            x + 0 = \surr{x^L}{x^R} = x.
        \]
    \end{example}

    \begin{example}
        Tambi\'en hemos definido los n\'umeros $1$ y $-1$. Miremos qu\'e pasa cuando se suman entre ellos.

        Primero,
        \[
            1 + (-1) = \surr{1^L + (-1)}{1+(-1)^R} = \surr{0+(-1)}{1+0} = \surr{-1}{1} = 0.
        \]

        Tambi\'en tenemos que
        \[
            1 + 1 = \surr{1^L+1, 1+1^L}{} = \surr{0+1,1+0}{} = \surr{1}{},
        \]
        por lo tanto, llamamos al n\'umero surreal $\surr{1}{} = 2$. Un an\'alisis parecido se puede hacer para decir que $(-1)+(-1) = \surr{}{-1}$, por lo tanto $-2 = \surr{}{-1}$.
    \end{example}

    La operaci\'on de suma en los n\'umeros reales genera un grupo conmutativo, en nuestro caso, ya hemos mostrado que la suma es conmutativa y que adem\'as tiene un m\'odulo, lo que nos falta para mostrar que la suma en los n\'umeros surreales genera un grupo conmutativo es mostrar que todos los elementos tienen inversos aditivos y adem\'as que la suma es asociativa.

    \begin{theorem}[Asociatividad de la suma]
        Sean $x, y, z$ n\'umeros surreales. Tenemos que
        \[
            (x+y)+z = x+(y+z).
        \]
    \end{theorem}

    \begin{proof}
        Veamos el teorema por inducci\'on. Nuestro caso base es cuando alguno todos los elementos son $0$, en este caso tenemos
        \[
            (0+0)+0 = 0+0 = 0 = 0+0 = 0+(0+0).
        \]

        Ahora, nuestra hip\'otesis de inducci\'on es que la asociatividad se cumple para los elementos de los conjuntos $L$ y $R$ de $x,y,z$, por ejemplo, de los elementos de los conjuntos $L$ se tiene que
        \begin{align*}
            (x^L+y)+z = x^L + (y+z), \\
            (x+y^L)+z = x + (y^L+z), \\
            (x+y)+z^L = x + (y+z^L).
        \end{align*}

        Entonces tenemos que
        \begin{align*}
            (x+y)+z & = \surr{(x+y)^L+z, (x+y)+z^L}{\dots} \\
                    & = \surr{(x^L+y)+z, (x+y^L)+z, (x+y)+z^L}{\dots} \\
                    & = \surr{x^L+(y+z), x+(y^L+z), x+(y+z^L)}{\dots} \\
                    & = \surr{x^L+(y+z), x+(y+z)^L}{\dots} \\
                    & = x+(y+z).
        \end{align*}
        La demostraci\'on del conjunto $R$ se hace de la misma manera.
    \end{proof}

    Los inversos aditivos vamos a definirlos tambi\'en recursivamente, en base a los inversos aditivos de los elementos de $L$ y $R$ del n\'umero.

    \begin{definition}[Inversos aditivos]
        Sea $x$ un n\'umero surreal. Definimos su inverso aditivo como
        \[
            (-x) = \surr{-(x^L)}{-(x^R)}.
        \]
    \end{definition}

    \begin{example}
        Veamos los inversos de los n\'umeros que ya nombramos. Por un lado tenemos que
        \[
            -0 = \surr{-0^R}{-0^L} = \surr{}{} = 0,
        \]
        puesto que sus conjuntos $L$ y $R$ son vac\'ios.

        Veamos tambi\'en que efectivamente aquel que llamamos $-1$ en la secci\'on anterior es en efecto el inverso aditivo de $1$,
        \[
            -1 = \surr{-1^R}{-1^L} = \surr{}{-0} = \surr{}{0} = -1.     
        \]

        Tambi\'en podemos hacer el mismo chequeo para $2$ y $-2$.
    \end{example}

    \begin{theorem}
        Sea $x$ un n\'umero surreal. Tenemos que $x+(-x)=0$.
    \end{theorem}

    \begin{proof}
        Mostremos el teorema por inducci\'on. Primero miremos el caso base cuando $x=0$. Tenemos que
        \[
            x + (-x) = 0 + (-0) = 0+0 = 0,
        \]
        por lo tanto se cumple.

        Ahora, supongamos que es verdad para los elementos de los conjuntos $L$ y $R$ de $x$. Mostremos que $x+(-x) \le 0$ por contradicci\'on, es decir, supongamos que es mentira luego existe alg\'un elemento tal que $(x+(-x))^R < 0$. Por lo tanto, o $x^R + (-x) < 0$ o $x+(-x^L) < 0$, pero en los conjuntos $L$ de estos n\'umeros est\'an los elementos $x^R + (-x^R)$ y $x^L + (-x^L)$ respectivamente, que cumplen  
        $x^R + (-x^R) \ge 0$ y $x^L + (-x^L)\ge 0$ por hip\'otesis de inducci\'on, lo cual contradice que $x^R + (-x) < 0$ o $x+(-x^L) < 0$.

        Para mostrar que $x+(-x) \ge 0$ el proceso es similar.
    \end{proof}


    F\'ijese que en los ejemplos tenemos que $1+(-1) = 0$ pero los conjuntos son diferentes, ya que $0=\surr{}{}$ mientras que $1+(-1) = \surr{-1}{1}$. A\'un no hemos mostrado si la suma es compatible con estas clases de equivalencia, en otras palabras, queremos mostrar que si $x = x'$ entonces $x+y = x'+y$ para todo $y$ n\'umero surreal de modo que no importe cual representante de la clase se utilice en las sumas. Para hacer esto vamos a primero probarlo para la relaci\'on de orden, ya que si lo mostramos para la relaci\'on de orden entonces lo tendremos tambi\'en para la relaci\'on de equivalencia por definici\'on.

    Mostremoslo primero para el elemento m\'as simple, es decir, el $0$.

    \begin{theorem}
        Si tenemos $0 \le x$, entonces $y \le x + y$ para todo n\'umero surreal $y$.
    \end{theorem}

    \begin{proof}
        Vamos a probar el teorema por inducci\'on. Los casos bases de nuestra inducci\'on ser\'an cuando $y=0$ o cuando $x=0$, en cualquiera de los dos casos la proposici\'on es verdadera.

        Supondremos como hip\'otesis de inducci\'on que la proposici\'on es verdad para los elementos de $L$ y $R$ de $x$ y $y$, esto es,
        \begin{align*}
            0\le x & \implies y^L \le x + y^L, \\
            0\le x & \implies y^R \le x + y^R, \\
            0\le x^R & \implies y \le x^R + y, \\
            0\le x^L & \implies y \le x^L + y.
        \end{align*}

        Ahora, supongamos que $0\le x$. Queremos mostrar que $y \le x+y$, esto es, $y^L < x+y$ y $y < (x+y)^R$. F\'ijese que en el conjunto $(x+y)^L$ se encuentra $x+y^L$, y por hip\'otesis de inducci\'on tenemos que $y^L \le x+y^L < (x+y)$, por lo tanto la primera parte se cumple.
        
        Ahora, $(x+y)^R$ puede ser o $x+y^R$ o $x^R+y$. Supongamos que es de la forma $x+y^R$. En este caso tenemos que $y < y^R \le x+y^R$. Por otro lado, si el elemento es de la forma $x^R+y$ entonces por  hip\'otesis de inducci\'on tenemos que $y \le x^R+y$, la desigualdad es estricta pues como $x^R > 0$ entonces existe alg\'un elemento $(x^R)^L \ge 0$ y por hip\'otesis de inducci\'on $(x^R)^L + y \ge y$, pero f\'ijese que $(x^R)^L + y$ est\'a en el conjunto $L$ de $x^R+y$ por lo tanto tendremos que $y \le (x^R)^L + y < x^R+y$, con lo que probamos que $y < (x+y)^R$.
    \end{proof}

    \begin{corollary}
        \label{zero-sum}
        Si tenemos $x = 0$, entonces $y = y+x$.
    \end{corollary}

    \begin{theorem}
        Si tenemos que $x+y\le x+z$ entonces $y\le z$.
    \end{theorem}

    \begin{proof}
        Haremos esta demostraci\'on por inducci\'on. El caso base es cuando $y=z=0$, y este se puede verificar que la proposici\'on se cumple.

        Como hip\'otesis de inducci\'on supondremos que se cumple para todo los elementos de $L$ y $R$ de $z$ y $y$ respectivamente, esto es,
        \begin{align*}
            x+y\le x+z^R & \implies y\le z^R,\\
            x+y^L\le x+z & \implies y^L\le z, \\
            x^R+y\le x^R+z & \implies y\le z,\\
            x^L+y\le x^L+z & \implies y\le z.
        \end{align*}

        Supongamos que $x+y\le x+z$, entonces $x+y^L < x+z$ y $x+y < x+z^R$. 
        
        Utilicemos la primera desigualdad, como es una desigualdad estricta entonces tendremos que existe alg\'un 
        \begin{equation}
            \label{rec-sum-ord}
            (x+y^L)^R \le x+z,
        \end{equation}
        estos elementos puedes ser o $x+(y^L)^R$ o $x^R+y^L$, en el primer caso tendremos por hip\'otesis de inducci\'on que $(y^L)^R \le z$ luego $y^L < z$. Por otro lado, si el elemento es de la forma $x^R+y^L$ entonces tenemos la desigualdad $x^R+y^L \le x+z < x^R+z$, de la que podemos deducir que existe un elemento $(x^R+y^L)^R \le x^R+z$ la cual es muy parecida a la ecuaci\'on \ref{rec-sum-ord} por lo tanto podremos repetir el mismo argumento hasta que el elemento sea de la forma $\left((x^R)^{\dots}\right)^R + (y^L)^R$ y hacer el mismo el mismo argumento para deducir que $y^L < z$.

        Con la desigualdad $x+y < x+z^R$ se trabaja de manera similar y se deduce que $y < z^R$, por lo tanto, $y\le z$.
    \end{proof}

    \begin{corollary}
        Si $y \le z$, entonces $y+x \le z+x$.
    \end{corollary}

    \begin{proof}
        Supongamos que $y\le z$. Utilizando el corolario \ref{zero-sum}, tenemos que $y+x+(-x)\le z+x+(-x)$ puesto que le estamos sumando $0$ a ambos lados. Ahora, usando el teorema anterior podemos cancelar a ambos lados de la desigualdad y tenemos que $y+x\le z+x$.
    \end{proof}

    \begin{corollary}
        Si $y=z$, entonces $y+x=z+x$.
    \end{corollary}

    La multiplicacion de numeros tambien se define recursivamente, se dice que
    \[
        xy = \{x^Ly+xy^L-x^Ly^L, x^Ly+xy^R-x^Ry^R| x^Ly+xy^R-x^Ly^R, x^Ry+xy^L-x^Ry^L\}.
    \]

    Esta definicion un tanto mas compleja que la de la suma sale de la multiplicacion de las desigualdades
    \begin{align*}
        (x-x^L) > 0, \quad & (x^R-x)>0 \\
        (y-y^L) > 0, \quad & (y^R-y)>0.
    \end{align*}
    