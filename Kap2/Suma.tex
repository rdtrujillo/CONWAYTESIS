\section{Suma de números surreales}

    La suma de los números surreales también se define recursivamente.
    
    \begin{definition}[Definici\'on de suma]
        Se dice que
        \[
            x + y  = \left\{x^L+y, x+y^L\;|\;x^R+y, x+y^R\right\}.
        \]
        Esta definici\'on hace que la suma sea autom\'aticamente conmutativa ya que es lo mismo en definici\'on hacer $x+y$ o $y+x$.
    \end{definition}

    Como todas las definiciones recursivas que hemos hecho, la suma se define eventualmente en base al número $0$, por eso es bueno ver qué pasa cuando se suma $0$.

    \begin{example}
        ¿Qué pasa cuando se suma $0+0$? Sabemos que $0 = \surr{}{}$, por lo tanto, no existe ni $0^L$ ni $0^R$. Lo que tendremos entonces es
        \[
            0+0 = \surr{0^L+0, 0+0^L}{0^R+0, 0+0^R} = \surr{}{} = 0,
        \]
        por lo tanto tendremos que $0+0 = 0$.
    \end{example}

    \begin{example}
        Más aún, si el $0$ de los números surreales se parece al $0$ de los números reales entonces tendríamos que el $0$ es módulo de la suma, esto es, $x+0 = x$ para todo $x$ n\'umero surreal.

        Mostremos esto por inducci\'on. Nuestro caso base es cuando $x = 0$, es decir, $0+0$ que ya probamos que $0+0=0=x$ luego se cumple.

        Supongamos que se cumple para todos los elementos en los conjuntos $X^L$ y $X^R$. Tenemos entonces que
        \[
            x + 0 = \surr{x^L+0, x+0^L}{x^R+0, x+0^L} = \surr{x^L+0}{x^R+0}
        \]
        y por hip\'otesis de inducci\'on tenemos que $x^L+0 = x^L$ y $x^R+0 = x^R$, por lo tanto 
        \[
            x + 0 = \surr{x^L}{x^R} = x.
        \]
    \end{example}

    \begin{example}
        Tambi\'en hemos definido los n\'umeros $1$ y $-1$. Miremos qu\'e pasa cuando se suman entre ellos.

        Primero,
        \[
            1 + (-1) = \surr{1^L + (-1)}{1+(-1)^R} = \surr{0+(-1)}{1+0} = \surr{-1}{1} = 0.
        \]

        Tambi\'en tenemos que
        \[
            1 + 1 = \surr{1^L+1, 1+1^L}{} = \surr{0+1,1+0}{} = \surr{1}{},
        \]
        por lo tanto, llamamos al n\'umero surreal $\surr{1}{} = 2$. Un an\'alisis parecido se puede hacer para decir que $(-1)+(-1) = \surr{}{-1}$, por lo tanto $-2 = \surr{}{-1}$.
    \end{example}

    La operaci\'on de suma en los n\'umeros reales genera un grupo conmutativo, en nuestro caso, ya hemos mostrado que la suma es conmutativa y que adem\'as tiene un m\'odulo, lo que nos falta para mostrar que la suma en los n\'umeros surreales genera un grupo conmutativo es mostrar que todos los elementos tienen inversos aditivos y adem\'as que la suma es asociativa.

    \begin{theorem}[Asociatividad de la suma]
        Sean $x, y, z$ n\'umeros surreales. Tenemos que
        \[
            (x+y)+z = x+(y+z).
        \]
    \end{theorem}

    \begin{proof}
        Veamos el teorema por inducci\'on. Nuestro caso base es cuando alguno todos los elementos son $0$, en este caso tenemos
        \[
            (0+0)+0 = 0+0 = 0 = 0+0 = 0+(0+0).
        \]

        Ahora, nuestra hip\'otesis de inducci\'on es que la asociatividad se cumple para los elementos de los conjuntos $L$ y $R$ de $x,y,z$, por ejemplo, de los elementos de los conjuntos $L$ se tiene que
        \begin{align*}
            (x^L+y)+z = x^L + (y+z), \\
            (x+y^L)+z = x + (y^L+z), \\
            (x+y)+z^L = x + (y+z^L).
        \end{align*}

        Entonces tenemos que
        \begin{align*}
            (x+y)+z & = \surr{(x+y)^L+z, (x+y)+z^L}{\dots} \\
                    & = \surr{(x^L+y)+z, (x+y^L)+z, (x+y)+z^L}{\dots} \\
                    & = \surr{x^L+(y+z), x+(y^L+z), x+(y+z^L)}{\dots} \\
                    & = \surr{x^L+(y+z), x+(y+z)^L}{\dots} \\
                    & = x+(y+z).
        \end{align*}
        La demostraci\'on del conjunto $R$ se hace de la misma manera.
    \end{proof}

    Los inversos aditivos vamos a definirlos tambi\'en recursivamente, en base a los inversos aditivos de los elementos de $L$ y $R$ del n\'umero.

    \begin{definition}[Inversos aditivos]
        Sea $x$ un n\'umero surreal. Definimos su inverso aditivo como
        \[
            (-x) = \surr{-(x^L)}{-(x^R)}.
        \]
    \end{definition}

    \begin{example}
        Veamos los inversos de los n\'umeros que ya nombramos. Por un lado tenemos que
        \[
            -0 = \surr{-0^R}{-0^L} = \surr{}{} = 0,
        \]
        puesto que sus conjuntos $L$ y $R$ son vac\'ios.

        Veamos tambi\'en que efectivamente aquel que llamamos $-1$ en la secci\'on anterior es en efecto el inverso aditivo de $1$,
        \[
            -1 = \surr{-1^R}{-1^L} = \surr{}{-0} = \surr{}{0} = -1.     
        \]

        Tambi\'en podemos hacer el mismo chequeo para $2$ y $-2$.
    \end{example}

    \begin{theorem}
        Sea $x$ un n\'umero surreal. Tenemos que $x+(-x)=0$.
    \end{theorem}

    \begin{proof}
        Mostremos el teorema por inducci\'on. Primero miremos el caso base cuando $x=0$. Tenemos que
        \[
            x + (-x) = 0 + (-0) = 0+0 = 0,
        \]
        por lo tanto se cumple.

        Ahora, supongamos que es verdad para los elementos de los conjuntos $L$ y $R$ de $x$. Mostremos que $x+(-x) \le 0$ por contradicci\'on, es decir, supongamos que es mentira luego existe alg\'un elemento tal que $(x+(-x))^R < 0$. Por lo tanto, o $x^R + (-x) < 0$ o $x+(-x^L) < 0$, pero en los conjuntos $L$ de estos n\'umeros est\'an los elementos $x^R + (-x^R)$ y $x^L + (-x^L)$ respectivamente, que cumplen  
        $x^R + (-x^R) \ge 0$ y $x^L + (-x^L)\ge 0$ por hip\'otesis de inducci\'on, lo cual contradice que $x^R + (-x) < 0$ o $x+(-x^L) < 0$.

        Para mostrar que $x+(-x) \ge 0$ el proceso es similar.
    \end{proof}


    F\'ijese que en los ejemplos tenemos que $1+(-1) = 0$ pero los conjuntos son diferentes, ya que $0=\surr{}{}$ mientras que $1+(-1) = \surr{-1}{1}$. A\'un no hemos mostrado si la suma es compatible con estas clases de equivalencia, en otras palabras, queremos mostrar que si $x = x'$ entonces $x+y = x'+y$ para todo $y$ n\'umero surreal de modo que no importe cual representante de la clase se utilice en las sumas. Para hacer esto vamos a primero probarlo para la relaci\'on de orden, ya que si lo mostramos para la relaci\'on de orden entonces lo tendremos tambi\'en para la relaci\'on de equivalencia por definici\'on.

    Mostremoslo primero para el elemento m\'as simple, es decir, el $0$.

    \begin{theorem}
        Si tenemos $0 \le x$, entonces $y \le x + y$ para todo n\'umero surreal $y$.
    \end{theorem}

    \begin{proof}
        Vamos a probar el teorema por inducci\'on. Los casos bases de nuestra inducci\'on ser\'an cuando $y=0$ o cuando $x=0$, en cualquiera de los dos casos la proposici\'on es verdadera.

        Supondremos como hip\'otesis de inducci\'on que la proposici\'on es verdad para los elementos de $L$ y $R$ de $x$ y $y$, esto es,
        \begin{align*}
            0\le x & \implies y^L \le x + y^L, \\
            0\le x & \implies y^R \le x + y^R, \\
            0\le x^R & \implies y \le x^R + y, \\
            0\le x^L & \implies y \le x^L + y.
        \end{align*}

        Ahora, supongamos que $0\le x$. Queremos mostrar que $y \le x+y$, esto es, $y^L < x+y$ y $y < (x+y)^R$. F\'ijese que en el conjunto $(x+y)^L$ se encuentra $x+y^L$, y por hip\'otesis de inducci\'on tenemos que $y^L \le x+y^L < (x+y)$, por lo tanto la primera parte se cumple.
        
        Ahora, $(x+y)^R$ puede ser o $x+y^R$ o $x^R+y$. Supongamos que es de la forma $x+y^R$. En este caso tenemos que $y < y^R \le x+y^R$. Por otro lado, si el elemento es de la forma $x^R+y$ entonces por  hip\'otesis de inducci\'on tenemos que $y \le x^R+y$, la desigualdad es estricta pues como $x^R > 0$ entonces existe alg\'un elemento $(x^R)^L \ge 0$ y por hip\'otesis de inducci\'on $(x^R)^L + y \ge y$, pero f\'ijese que $(x^R)^L + y$ est\'a en el conjunto $L$ de $x^R+y$ por lo tanto tendremos que $y \le (x^R)^L + y < x^R+y$, con lo que probamos que $y < (x+y)^R$.
    \end{proof}

    \begin{corollary}
        \label{zero-sum}
        Si tenemos $x = 0$, entonces $y = y+x$.
    \end{corollary}

    \begin{theorem}
        Si tenemos que $x+y\le x+z$ entonces $y\le z$.
    \end{theorem}

    \begin{proof}
        Haremos esta demostraci\'on por inducci\'on. El caso base es cuando $y=z=0$, y este se puede verificar que la proposici\'on se cumple.

        Como hip\'otesis de inducci\'on supondremos que se cumple para todo los elementos de $L$ y $R$ de $z$ y $y$ respectivamente, esto es,
        \begin{align*}
            x+y\le x+z^R & \implies y\le z^R,\\
            x+y^L\le x+z & \implies y^L\le z, \\
            x^R+y\le x^R+z & \implies y\le z,\\
            x^L+y\le x^L+z & \implies y\le z.
        \end{align*}

        Supongamos que $x+y\le x+z$, entonces $x+y^L < x+z$ y $x+y < x+z^R$. 
        
        Utilicemos la primera desigualdad, como es una desigualdad estricta entonces tendremos que existe alg\'un 
        \begin{equation}
            \label{rec-sum-ord}
            (x+y^L)^R \le x+z,
        \end{equation}
        estos elementos puedes ser o $x+(y^L)^R$ o $x^R+y^L$, en el primer caso tendremos por hip\'otesis de inducci\'on que $(y^L)^R \le z$ luego $y^L < z$. Por otro lado, si el elemento es de la forma $x^R+y^L$ entonces tenemos la desigualdad $x^R+y^L \le x+z < x^R+z$, de la que podemos deducir que existe un elemento $(x^R+y^L)^R \le x^R+z$ la cual es muy parecida a la ecuaci\'on \ref{rec-sum-ord} por lo tanto podremos repetir el mismo argumento hasta que el elemento sea de la forma $\left((x^R)^{\dots}\right)^R + (y^L)^R$ y hacer el mismo el mismo argumento para deducir que $y^L < z$.

        Con la desigualdad $x+y < x+z^R$ se trabaja de manera similar y se deduce que $y < z^R$, por lo tanto, $y\le z$.
    \end{proof}

    \begin{corollary}
        Si $y \le z$, entonces $y+x \le z+x$.
    \end{corollary}

    \begin{proof}
        Supongamos que $y\le z$. Utilizando el corolario \ref{zero-sum}, tenemos que $y+x+(-x)\le z+x+(-x)$ puesto que le estamos sumando $0$ a ambos lados. Ahora, usando el teorema anterior podemos cancelar a ambos lados de la desigualdad y tenemos que $y+x\le z+x$.
    \end{proof}

    \begin{corollary}
        Si $y=z$, entonces $y+x=z+x$.
    \end{corollary}